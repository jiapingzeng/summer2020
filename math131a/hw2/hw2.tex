\documentclass{article}
\usepackage[margin=1in]{geometry}
\usepackage{enumitem}
\usepackage{setspace}
\usepackage{amsmath}
\usepackage{amssymb}

\title{Math 131A Homework 2}
\date{8/12/2020}
\author{Jiaping Zeng}

\begin{document}
\setstretch{1.35}
\maketitle

\begin{itemize}
    \item [8.1] Prove the following:
          \begin{itemize}
              \item [(a)] lim $\dfrac{(-1)^n}{n}=0$
              \item [(c)] lim $\dfrac{2n-1}{3n+2}=\dfrac{2}{3}$
          \end{itemize}
    \item [8.4] Let $(t_n)$ be a bounded sequence, i.e., there exists $M$ such that $|t_n|\leq M$ for all $n$, and let $(s_n)$ be a sequence such that lim $s_n=0$. Prove lim $(s_nt_n)=0$.
    \item [8.5]
          \begin{itemize}
              \item [(a)] Consider three sequences $(a_n)$, $(b_n)$ and $(s_n)$ such that $a_n\leq s_n\leq b_n$ for all $n$ and lim $a_n=$ lim $b_n=s$. Prove lim $s_n=s$. This is called the ``squeeze lemma.''
              \item [(b)] Suppose $(s_n)$ and $(t_n)$ are sequences such that $|s_n|\leq t_n$ for all $n$ and lim $t_n=0$. Prove lim $s_n=0$.
          \end{itemize}
    \item [8.6] Let $(s_n)$ be a sequence in $\mathbb{R}$.
          \begin{itemize}
              \item [(a)] Prove lim $s_n=0$ if and only if lim $|s_n|=0$.
              \item [(b)] Observe that if $s_n=(-1)^n$, then lim $|s_n|$ exists, but lim $s_n$ does not exist.
          \end{itemize}
    \item [8.9] Let $(s_n)$ be a sequence that converges.
          \begin{itemize}
              \item [(a)] Show that if $s_n\geq a$ for all but finitely many $n$, then lim $s_n\geq a$.
              \item [(b)] Show that if $s_n\leq b$ for all but finitely many $n$, then lim $s_n\leq b$.
              \item [(c)] Conclude that if all but finitely many $s_n$ belong to $[a,b]$, then lim $s_n$ belongs to $[a,b]$.
          \end{itemize}
    \item [8.10] Let $(s_n)$ be a convergent sequence, and suppose lim $s_n>a$. Prove there exists a number $N$ such that $n>N$ implies $s_n>a$.
    \item [9.1]
          \begin{itemize}
              \item [(a)]
              \item [(b)]
          \end{itemize}
    \item [9.3]
    \item [9.9]
    \item [9.10]
          \begin{itemize}
              \item [(a)]
              \item [(b)]
          \end{itemize}
    \item [9.11]
    \item [9.12]
    \item [10.5]
    \item [10.6]
    \item [10.7]
    \item [P1]
\end{itemize}
\end{document}