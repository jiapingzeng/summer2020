\documentclass{article}
\usepackage[margin=1in]{geometry}
\usepackage{enumitem}
\usepackage{setspace}
\usepackage{amsmath}
\usepackage{amssymb}
\usepackage{physics}

\title{Math 131A Homework 2}
\date{8/12/2020}
\author{Jiaping Zeng}

\begin{document}
\setstretch{1.35}
\maketitle

\begin{itemize}
      \item [8.1] Prove the following:
            \begin{itemize}
                  \item [(a)] $\text{lim }\dfrac{(-1)^n}{n}=0$\\
                        \textbf{Answer: } Proof: Take $N=\dfrac{1}{\epsilon}$, then $n>N=\dfrac{1}{\epsilon}\implies\epsilon>\dfrac{1}{n}=\abs{s_n-0}$. Therefore $\text{lim }\dfrac{(-1)^n}{n}=0$ by definition of limit.
                  \item [(c)] $\text{lim }\dfrac{2n-1}{3n+2}=\dfrac{2}{3}$\\
                        \textbf{Answer: }\\
                        Scratch work: $\abs{\dfrac{2n-1}{3n+2}-\dfrac{2}{3}}<\epsilon\implies\abs{\dfrac{6n-3-2\cdot(3n+2)}{3\cdot(3n+2)}}<\epsilon\implies\abs{\dfrac{-7}{3\cdot(3n+2)}}<\epsilon\implies\dfrac{7}{3\epsilon}<3n+2\implies\dfrac{7}{9\epsilon}-\dfrac{2}{3}\leq n$.\\
                        Proof: Let $\epsilon>0$, define $N=\dfrac{7}{9\epsilon}-\dfrac{2}{3}$. Then, $n>N=\dfrac{7}{9\epsilon}-\dfrac{2}{3}\implies 3n+2>\dfrac{7}{3\epsilon}\implies 3\epsilon>\dfrac{7}{3n+2}\implies\epsilon>\dfrac{7}{3\cdot(3n+2)}=\abs{s_n-\dfrac{2}{3}}$. Therefore $\text{lim }\dfrac{2n-1}{3n+2}=\dfrac{2}{3}$ by definition.
            \end{itemize}
      \item [8.4] Let $(t_n)$ be a bounded sequence, i.e., there exists $M$ such that $|t_n|\leq M$ for all $n$, and let $(s_n)$ be a sequence such that $\text{lim }s_n=0$. Prove $\text{lim }(s_nt_n)=0$.\\
            \textbf{Answer: } Since $\text{lim }s_n=0$, there exists an $N$ such that $n>N\implies\abs{s_n}<\epsilon$. Then, $\abs{s_n}\cdot\abs{t_n}<\epsilon\cdot\abs{t_n}\implies\abs{s_nt_n}<M\epsilon$ (multiply by $\abs{t_n}$; assuming that $\abs{t_n}\neq 0$, since clearly $\text{lim }s_nt_n=0$ when $\abs{t_n}=0$). Since $\epsilon$ is an arbitrary constant, we can take another $\epsilon_2=\dfrac{\epsilon}{M}>0$, which exists since $\mathbb{Q}$ is dense in $\mathbb{R}$. Then $\abs{s_nt_n}<\epsilon_2$ and $\text{lim }(s_nt_n)=0$ by definition of limit.
      \item [8.5]
            \begin{itemize}
                  \item [(a)] Consider three sequences $(a_n)$, $(b_n)$ and $(s_n)$ such that $a_n\leq s_n\leq b_n$ for all $n$ and $\text{lim }a_n=$ $\text{lim }b_n=s$. Prove $\text{lim }s_n=s$. This is called the ``squeeze lemma.''\\
                        \textbf{Answer: } Since $\text{lim }a_n=s$, there exists an $N_a$ such that $n>N_a\implies\abs{a_n-s}<\epsilon\implies -\epsilon<a_n-s<\epsilon \implies s-\epsilon<a_n<s+\epsilon$. Similarly, since $\text{lim }b_n=s$, there exists an $N_b$ such that $n>N_b\implies\abs{b_n-s}<\epsilon\implies s-\epsilon<b_n<s+\epsilon$. Then, since $a_n\leq s_n\leq b_n$ for all $n$, we also have $s-\epsilon<a_n\leq s_n\leq b_n<s+\epsilon$. Therefore $s-\epsilon<s_n<s+\epsilon$, which implies that $\abs{s_n-s}<\epsilon$. Then $\text{lim }s_n=s$ by definition of limit.
                  \item [(b)] Suppose $(s_n)$ and $(t_n)$ are sequences such that $|s_n|\leq t_n$ for all $n$ and $\text{lim }t_n=0$. Prove $\text{lim }s_n=0$.\\
                        \textbf{Answer: } Since $\text{lim }t_n=0$, there exists an $N$ such that $n>N\implies \abs{t_n}<\epsilon$. Then we also know that $\text{lim }-t_n=0$ because $\abs{-t_n}=\abs{t_n}<\epsilon$. Then, since $\abs{s_n}\leq t_n$, we have $-t_n\leq s_n\leq t_n$. Therefore $\text{lim }s_n=0$ by squeeze lemma.
            \end{itemize}
      \item [8.6] Let $(s_n)$ be a sequence in $\mathbb{R}$.
            \begin{itemize}
                  \item [(a)] Prove $\text{lim }s_n=0$ if and only if $\text{lim }|s_n|=0$.\\
                        \textbf{Answer: }
                        \begin{itemize}
                              \item [$\Rightarrow$:] Since $\text{lim }s_n=0$, there exists an $N$ such that $n>N\implies\abs{s_n}<\epsilon$. Then $\text{lim }\abs{s_n}=0$ by definition because $\abs{(\abs{s_n})}=\abs{s_n}<\epsilon$.
                              \item [$\Leftarrow$:] Since $\text{lim }\abs{s_n}=0$, there exists an $N$ such that $n>N\implies\abs{(\abs{s_n})}<\epsilon$. Since $\abs{(\abs{s_n})}=\abs{s_n}$, we also have $\abs{s_n}<\epsilon$. Then $\text{lim }s_n=0$ by definition.
                        \end{itemize}
                  \item [(b)] Observe that if $s_n=(-1)^n$, then $\text{lim }|s_n|$ exists, but $\text{lim }s_n$ does not exist.\\
                        \textbf{Answer: } $\text{lim }\abs{s_n}=$ $\text{lim }\abs{(-1)^n}=$ $\text{lim }1$. We will show that $\text{lim }s_n$ does not exist by contradiction. Assume $\text{lim }s_n=s$ exists, then there exists an $N$ such that $n>N\implies\abs{(-1)^n-s}<\epsilon$. We know that $\abs{(-1)^n-s}=\abs{-1-s}$ for odd $n$, $\abs{1-s}$ for even $n$, and $\abs{-1-s}\neq\abs{1-n}$, then one of them must be nonzero. Then there would exist an $\epsilon>0$ such that max$\{\abs{-1-s},\abs{1-s}\}\geq\epsilon$, which contradicts with $\text{lim }s_n$ exists. Therefore $\text{lim }s_n$ does not exist by contradiction.
            \end{itemize}
      \item [8.9] Let $(s_n)$ be a sequence that converges.
            \begin{itemize}
                  \item [(a)] Show that if $s_n\geq a$ for all but finitely many $n$, then $\text{lim }s_n\geq a$.\\
                        \textbf{Answer: } By contradiction. Since $s_n\geq a$ for all but finitely many $n$, there exists an $N_0$ such that $s_n\geq a$ for all $n>N_0$. Then, let $s=\text{lim }s_n$ and assume $s<a$, then pick $\epsilon=a-s>0$. Select $N\geq N_0$, then by definition of limit, we have $n>N\implies\abs{s_n-s}<\epsilon=a-s$. By expanding the previous expression we have $-a+s<s_n-s<a-s\implies s_n-s<a-s\implies s_n<a$, whcih contradicts with $s_n\geq a$ for $n>N_0$. Therefore $s=\text{lim }s_n\geq a$ by contradiction.
                  \item [(b)] Show that if $s_n\leq b$ for all but finitely many $n$, then $\text{lim }s_n\leq b$.\\
                        \textbf{Answer: } Again by contradiction. Since $s_n\leq b$ for all but finitely many $n$, there exists an $n>N_0$ such that $s_n\leq b$ for all $n>N_0$. Then, let $s=\text{lim }s_n$ and assume $s>b$, then pick $\epsilon=s-b>0$. Select $N\geq N_0$, we have $n>N\implies\abs{s_n-s}<\epsilon=s-b$. Expanding the absolute value results in $-s+b<s_n-s<s-b\implies -s+b<s_n-s\implies b<s_n$, which contradicts with $s_n\leq b$ for $n>N_0$. Therefore $s=\text{lim }s_n\leq b$ by contradiction.
                  \item [(c)] Conclude that if all but finitely many $s_n$ belong to $[a,b]$, then $\text{lim }s_n$ belongs to $[a,b]$.\\
                        \textbf{Answer: } By combining the previous parts, we have the following: if $a\leq s_n\leq b$ for finitely many $s_n$, then $a\leq \text{lim }s_n\leq b$. In other words, by using interval notations, if finitely many $s_n\in[a,b]$, then $\text{lim }\in[a,b]$.
            \end{itemize}
      \item [8.10] Let $(s_n)$ be a convergent sequence, and suppose $\text{lim }s_n>a$. Prove there exists a number $N$ such that $n>N$ implies $s_n>a$.\\
            \textbf{Answer: } Let $\text{lim }s_n=s$; since $s>a$, select $\epsilon=s-a>0$. Then there exists an $N$ such that $n>N\implies\abs{s_n-s}<\epsilon$. By expanding the absolute value we have $-\epsilon<s_n-s<\epsilon$, which is equivalent to $s-\epsilon<s_n<s+\epsilon\implies s-\epsilon<s_n$. By substituting $\epsilon=s-a$ we have $s-(s-a)<s_n\implies a<s_n$.
      \item [9.1] Prove the following:
            \begin{itemize}
                  \item [(a)] $\text{lim }\dfrac{n+1}{n}=1$\\
                        \textbf{Answer: } $\text{lim }\dfrac{n+1}{n}=\text{lim }(1+\dfrac{1}{n})$ by multiplying $\dfrac{1}{n}$ to both the numerator and the denominator. Then, $\text{lim }(1+\dfrac{1}{n})=\text{lim }1+\text{lim }\dfrac{1}{n}=1+0=1$ by sum limit law.
                  \item [(b)] $\text{lim }\dfrac{3n+7}{6n-5}=\dfrac{1}{2}$\\
                        \textbf{Answer: } $\text{lim }\dfrac{3n+7}{6n-5}=\text{lim }\dfrac{3+\frac{7}{n}}{6-\frac{5}{n}}$ (multiply by $\dfrac{\frac{1}{n}}{\frac{1}{n}}$). Then, in the numerator, we have $\text{lim }(3+\dfrac{7}{n})=\text{lim }3+\text{lim }\dfrac{7}{n}=3+0=3$ by sum limit law. Simiarly, in the denominator, we have $\text{lim }(6-\dfrac{5}{n})=\text{lim }6-\text{lim }\dfrac{5}{n}=6-0=6$. Then $\text{lim }\dfrac{3n+7}{6n-5}=\text{lim }\dfrac{3}{6}=\dfrac{1}{2}$.
            \end{itemize}
      \item [9.3] Suppose $\text{lim }a_n=a$, $\text{lim }b_n=b$, and $s_n=\dfrac{a_n^3+4a_n}{b_n^2+1}$. Prove $\text{lim }s_n=\dfrac{a^3+4a}{b^2+1}$, using the limit theorems.\\
            \textbf{Answer: } $\text{lim }s_n=\text{lim }\dfrac{a_n^3+4a_n}{b_n^2+1}$\\$=\text{lim }(a_n^3+4a_n)\cdot\text{lim }\dfrac{1}{b_n^2+1}$ (Product limit law)\\$=\dfrac{\text{lim }(a_n^3+4a_n)}{\text{lim }(b_n^2+1)}$ (Lemma 22)\\$=\dfrac{\text{lim }a_n^3+\text{lim }4a_n}{\text{lim }b_n^2+\text{lim }1}$ (Sum limit law)\\$=\dfrac{(\text{lim }a_n)^3+\text{lim }4a_n}{(\text{lim }b_n)^2+\text{lim }1}$ (Product limit law, applied to exponents)\\$=\dfrac{(\text{lim }a_n)^3+4\cdot\text{lim }a_n}{(\text{lim }b_n)^2+1}$ (Theorem 19)\\$=\dfrac{a^3+4a}{b^2+1}$ (Substitution)
      \item [9.9] Suppose there exists $N_0$ such that $s_n\leq t_n$ for all $n>N_0$.
            \begin{itemize}
                  \item [(a)] Prove that if $\text{lim }s_n=+\infty$, then $\text{lim }t_n=+\infty$.\\
                        \textbf{Answer: } Since $\text{lim }s_n=+\infty$, there exists an $N$ such that $n>N\implies s_n>M$ for all $M>0$. Then, since $t_n\geq s_n$ for $n>N_0$, we have $n>\text{max}\{N,N_0\}\implies t_n\geq s_n>M\implies t_n>M$. Therefore $\text{lim }t_n=+\infty$ by definition of divergence.
                  \item [(b)] Prove that if $\text{lim }t_n=-\infty$, then $\text{lim }s_n=-\infty$.\\
                        \textbf{Answer: } Similarly, since $\text{lim }t_n=-\infty$, there exists an $N$ such that $n>N\implies t_n<M$ for all $M<0$. Then, since $s_n\leq t_n$ for $n>N_0$, we have $n>\text{max}\{N,N_0\}\implies s_n\leq t_n<M\implies s_n<M$. Therefore $\text{lim }s_n=-\infty$ by definition of divergence.
                  \item [(c)] Prove that if $\text{lim }s_n$ and $\text{lim }t_n$ exist, then $\text{lim }s_n\leq\text{lim }t_n$.\\
                        \textbf{Answer: } Since we have $s_n\leq t_n$ for $n>N_0$, $t_n-s_n\geq 0$. Then $\text{lim }(t_n-s_n)\geq 0$ by 8.9(a). Furthermore, $\text{lim }t_n-\text{lim }s_n\geq 0$ by sum limit law; therefore $\text{lim }s_n\leq\text{lim }t_n$ upon rearranging the inequality.
            \end{itemize}
      \item [9.10]
            \begin{itemize}
                  \item [(a)] Show that if $\text{lim }s_n=+\infty$ and $k>0$, then lim$(ks_n)=+\infty$.\\
                        \textbf{Answer: } Since $\text{lim }s_n=+\infty$, there exists an $N$ such that $n>N\implies s_n>M$ for all $M>0$. Then we also have $n>N\implies ks_n>kM$ since $k>0$. We can then take $M_1=kM>0$ and $n>N\implies ks_n>M_1$ for all $M_1>0$, then $\text{lim}(ks_n)=+\infty$ by definition of divergence.
                  \item [(b)] Show $\text{lim }s_n=+\infty$ if and only if lim$(-s_n)=-\infty$.\\
                        \textbf{Answer: }
                        \begin{itemize}
                              \item [$\Rightarrow$:] Since $\text{lim }s_n=+\infty$, there exists an $N$ such that $n>N\implies s_n>M$ for all $M>0$. By multiplying $-1$ to both sides of the inequality (and flipping the inequality), we have $n>N\implies -s_n<-M$. Then let $M_1=-M$; we know $M_1<0$ since $M>0$. Then $n>N\implies -s_n<M_1$ for $M_1<0$, therefore $\text{lim}(-s_n)=-\infty$ by definition of divergence.
                              \item [$\Leftarrow$:] Similarly, since $\text{lim}(-s_n)=-\infty$, there exists an $N$ such that $n>N\implies -s_n<M$ for all $M<0$. Then we also have $s_n>-M$ by multiplying $-1$. Let $M_1=-M>0$, we have $n>N\implies s_n>M_1$, therefore $\text{lim }s_n=+\infty$.
                        \end{itemize}
            \end{itemize}
      \item [9.11]
            \begin{itemize}
                  \item [(a)] Show that if $\text{lim }s_n=+\infty$ and inf$\{t_n:n\in\mathbb{N}\}>-\infty$, then lim$(s_n+t_n)=+\infty$.\\
                  \textbf{Answer: } By definition of divergence, there exists an $N_s$ such that $n>N_s\implies s_n>M$ for all $M>0$. Additionally, we also have $t_n\geq\text{inf }t_n$ for all $n\in\mathbb{N}$ by definition of infimum. Then $n>N\implies s_n+t_n\geq s_n+x>M$.
                  \item [(b)] Show that if $\text{lim }s_n=+\infty$ and $\text{lim }t_n>-\infty$, then lim$(s_n+t_n)=+\infty$.\\
                  \textbf{Answer: } If $t_n$ diverges to $+\infty$, $\text{lim }s_n+t_n$ clearly diverges to $+\infty$ as well. If $t_n$ converges to $\text{lim }t_n=t\in\mathbb{R}$, there exists an $N_t$ such that $n>N_t\implies\abs{t_n-t}<\epsilon$ for $\epsilon>0\implies -\epsilon<t_n-t<\epsilon\implies t-\epsilon<t_n<t+\epsilon$. Since $s_n$ diverges to $+\infty$, we can pick $N=\text{max }(N_s,N_t)$ ($N_s$ same as part (a)); then $n>N\implies s_n+t_n>s_n+(t-\epsilon)>M$.
                  \item [(c)] Show that if $\text{lim }s_n=+\infty$ and if $(t_n)$ is a bounded sequence, then lim$(s_n+t_n)=+\infty$.\\
                  \textbf{Answer: } Similarly, since $(t_n)$ is bounded, then $n>N_s\implies s_n+t_n\geq s_n+\text{inf }t_n>M$.
            \end{itemize}
      \item [9.12] Assume all $s_n\neq 0$ and that the limit $L=$ $\text{lim }\abs{\dfrac{s_{n+1}}{s_n}}$ exists.
            \begin{itemize}
                  \item [(a)] Show that if $L<1$, then $\text{lim }s_n=0$.\\
                  \textbf{Answer: } Let $a\in\mathbb{R}$ such that $L<a<1$ and let $\epsilon=a-L>0$. Then since $L=\text{lim }\abs{\dfrac{s_{n+1}}{s_n}}$, there exists an $N$ such that $n>N\implies\abs{\abs{\dfrac{s_{n+1}}{s_n}}-L}<\epsilon\implies -a+L<\abs{\dfrac{s_{n+1}}{s_n}}-L<a-L\implies\abs{\dfrac{s_{n+1}}{s_n}}<a$. Therefore $\abs{s_{n+1}}<a\abs{s_n}$ for $n>N$. We will now show $\abs{s_n}<a^{n-N}\abs{s_N}$ for $n>N$ by induction as follows.\\
                  Base case ($n=N+1$): We want to show that $\abs{s_{n+1}}<a\abs{s_N}$, which is true as shown above.\\
                  Inductive step: Assume $\abs{s_n}<a^{n-N}\abs{s_N}$, we have $a\cdot\abs{s_n}<a\cdot a^{n-N}\abs{s_N}$ upon multiplying $a$ to both sides. Then, since $\abs{s_{n+1}}<a\abs{s_n}$ for $n>N$, $\abs{s_{n+1}}<a\cdot\abs{s_n}<a\cdot a^{n-N}\abs{s_N}=a^{(n+1)-N}\abs{s_N}$, which implies that $\abs{s_{n+1}}<a^{(n+1)-N}\abs{s_N}$. Therefore $\abs{s_n}<a^{n-N}\abs{s_N}$ by mathematical induction.\\
                  Since all $s_n\neq 0$, we have $0\leq\abs{s_n}\leq a^{n-N}\abs{s_N}$. In addition, we know that $\text{lim }a^{n-N}\abs{s_N}=0$ since $L<a<1$. Then $\text{lim }\abs{s_n}=0$ by squeeze lemma and therefore $\text{lim }s_n=0$.
                  \item [(b)] Show that if $L>1$, then $\text{lim }\abs{s_n}=+\infty$.\\
                  \textbf{Answer: } Let $t_n=\dfrac{1}{s_n}$, then $\text{lim }\abs{\dfrac{t_{n+1}}{t_n}}=\text{lim }\abs{\dfrac{\frac{1}{s_{n+1}}}{\frac{1}{s_n}}}=\text{lim }\abs{\dfrac{s_n}{s_{s_{n+1}}}}=\dfrac{1}{L}$. Since $L>1$, $\dfrac{1}{L}<1$; then $\text{lim }t_n=0$ by part (a). Therefore $\text{lim }\abs{\dfrac{1}{s_n}}=\text{lim }\abs{t_n}=\text{lim }t_n$ and $\text{lim }\abs{s_n}=+\infty$ by Theorem 9.10.
            \end{itemize}
      \item [10.5] Prove Theorem 10.4(ii).\\
      \textbf{Answer: } Let $(s_n)$ be an unbounded decreasing sequence and let $M<0$. Then, since $(s_n)$ is decreasing, it must be bounded above by $s_1$; since it is also unbounded, it must be unbounded below, i.e. $s_N<M$ for some $N$. Then we have $n>N\implies s_n\leq s_N<M$, therefore $\text{lim }s_n=-\infty$.
      \item [10.6] \begin{itemize}
                  \item [(a)] Let $(s_n)$ be a sequence such that \[\abs{s_{n+1}-s_n}<2^{-n} \text{ for all } n\in\mathbb{N}.\] Prove $(s_n)$ is a Cauchy sequence and hence a convergent sequence.\\
                  \textbf{Answer: }
                  \item [(b)] Is the result in $(a)$ true if we only assume $\abs{s_{n+1}-s_n}<\dfrac{1}{n}$ for all $n\in\mathbb{N}$?\\
                  \textbf{Answer: } No.
            \end{itemize}
      \item [10.7] Let $S$ be a bounded nonempty subset of $\mathbb{R}$ such that sup $S$ is not in $S$. Prove there is a sequence $(s_n)$ of points in $S$ such that $\text{lim }s_n=$ sup $S$.\\
      \textbf{Answer: } By definition of supremum, $\text{sup }S$ is the least upper bound; then we can take $m_n=\text{sup }S-\dfrac{1}{n}$ which cannot be an upper bound since $m<\text{sup }S$. Then, there must exist $s\in S$ such that $\text{sup }S>s_n>m_n$. Since $\text{sup }S$ is constant, $\text{lim sup }S=\text{sup }S$. In addition, $\text{lim }m_n=\text{lim sup }S-\text{lim }\dfrac{1}{n}=\text{sup }S-0=\text{sup }S$. Then we have $m<\text{sup }S$ with $\text{lim }m_n=\text{lim sup }S=\text{sup }S$; therefore $\text{lim }s_n=\text{sup }S$ by squeeze lemma.
      \item [P1] Let $(x_n)_{n\in\mathbb{N}}$ and $(y_n)_{n\in\mathbb{N}}$ be sequences of real numbers. Assume that $(x_n)_{n\in\mathbb{N}}$ is convergent and that the set \[\{n\in\mathbb{N}:x_n\neq y_n\}\] is finite. Prove that $(y_n)_{n\in\mathbb{N}}$ is also convergent and $\text{lim}_{n\rightarrow\infty}y_n=\text{lim}_{n\rightarrow\infty}x_n$.\\
      \textbf{Answer: } Let $\text{lim}_{n\rightarrow\infty}x_n=s$ and $\epsilon>0$. Since $(x_n)$ is convergent, there exists an $N_x$ such that $n>N_x\implies\abs{x_n-s}<\epsilon$. In addition, since the set $\{n\in\mathbb{N}:x_n\neq y_n\}$ is finite, there exists an $N_0$ such that $n>N_0\implies x_n=y_n$. Take $N=\text{max }\{N_0,N_x\}$ so that both are true, then $n>N\implies\abs{y_n-s}=\abs{x_n-s}<\epsilon$. Therefore  $\text{lim}_{n\rightarrow\infty}y_n=s=\text{lim}_{n\rightarrow\infty}x_n$.
\end{itemize}
\end{document}