\documentclass{article}
\usepackage[margin=1in]{geometry}
\usepackage{enumitem}
\usepackage{setspace}
\usepackage{amsmath}
\usepackage{amssymb}
\usepackage{physics}

\title{Math 131A Homework 4}
\date{8/27/2020}
\author{Jiaping Zeng}

\begin{document}
\setstretch{1.35}
\maketitle

\begin{itemize}
    \item [14.2] Determine which of the following series converge. Justify your answers.
          \begin{itemize}
              \item [(c)] $\sum\frac{3n}{n^3}=3\sum\frac{1}{n^2}$ which converges by p-test.
              \item [(d)] $\sum\frac{n^3}{3n}=\sum\frac{n^2}{3}$ which diverges since $\text{lim }\frac{n^2}{3}>0$.
              \item [(e)] $\sum\frac{n^2}{n!}=\sum\frac{n}{(n-1)!}\implies\abs{\frac{a_{n+1}}{a_n}}=\frac{n+1}{n!}\cdot\frac{(n-1)!}{n}=\frac{n+1}{n^2}\rightarrow 0<1$, therefore it converges by ratio test.
              \item [(f)] $\sum\frac{1}{n^n}\implies\sqrt[n]{\abs{a_n}}=\frac{1}{n}\rightarrow 0<1$, therefore it converges by root test.
          \end{itemize}
    \item [14.3]
          \begin{itemize}
              \item [(a)] $\sum\frac{1}{\sqrt{n!}}\implies\abs{\frac{a_{n+1}}{a_n}}=\frac{\sqrt{n!}}{\sqrt{(n+1)!}}=\sqrt{\frac{1}{n+1}}\rightarrow 0<1$, therefore it converges by ratio test.
              \item [(b)] $\sum\frac{2+\text{cos }n}{3^n}\implies\abs{\frac{a_{n+1}}{a_n}}=\abs{\frac{2+\text{cos}(n+1)}{3^{n+1}}\cdot\frac{3^n}{2+\text{cos }n}}=\abs{\frac{2+\text{cos}(n+1)}{6+\text{cos }n}}$; since $\abs{\text{cos }n}\leq 1$, $\abs{\frac{2+\text{cos}(n+1)}{6+\text{cos }n}}\leq\abs{\frac{2+1}{6-1}}=\frac{3}{5}<1$. Then $\text{lim sup }\abs{\frac{2+\text{cos}(n+1)}{6+\text{cos }n}}<1$ as well and the series converges by ratio test.
              \item [(c)] $\sum\frac{1}{2^n+n}\implies\abs{\frac{a_{n+1}}{a_n}}=\frac{2^n+n}{2^{n+1}+n+1}\rightarrow\frac{1}{2}<1$, therefore it converges by ratio test.
              \item [(d)] $\sum(\frac{1}{2})^n(50+\frac{2}{n})\implies\abs{\frac{a_{n+1}}{a_n}}=\frac{(\frac{1}{2})^{n+1}(50+\frac{2}{n+1})}{(\frac{1}{2})^n(50+\frac{2}{n})}=\frac{50+\frac{2}{n+1}}{100+\frac{4}{n}}\rightarrow\frac{1}{2}<1$, therefore it converges by ratio test.
          \end{itemize}
    \item [14.6]
          \begin{itemize}
              \item [(a)] Prove that if $\sum\abs{a_n}$ converges and $(b_n)$ is a bounded sequence, then $\sum a_nb_n$ converges.\\\textbf{Proof: } Since $\sum\abs{a_n}$ converges, there exists an $N$ such that $n\geq m>N\implies\abs{\sum_{k=m}^n \abs{a_k}}<\epsilon_a$ for $\epsilon_a>0$. In addition, since $(b_n)$ is bounded, there exists an $M>0$ such that $M\geq\abs{b_n}$ for all $n$. Then we can select $\epsilon=M\epsilon_a>0$, resulting in $n\geq m>N\implies M\abs{\sum_{k=m}^n \abs{a_k}}<M\epsilon_a=\epsilon$. Since $M\geq\abs{b_n}$, we have $\abs{\sum_{k=m}^n a_nb_n}\leq\abs{\sum_{k=m}^n \abs{a_kb_k}}\leq M\abs{\sum_{k=m}^n \abs{a_k}}$ by Exercise 3.6(b). Therefore $n\geq m>N\implies\abs{\sum_{k=m}^n a_nb_n}\leq\epsilon$ and $\sum a_nb_n$ converges by Theorem 14.4.
          \end{itemize}
    \item [14.7] Prove that if $\sum a_n$ is a convergent series of nonnegative numbers and $p>1$, then $\sum a_n^p$ converges.\\\textbf{Proof: } Since $\sum a_n$ is convergent, $\text{lim }a_n=0$ by Corollary 14.5. Then, let $\epsilon=1>0$, there exists an $N$ such that $n>N\implies a_n=\abs{a_n}<\epsilon$ ($a_n=\abs{a_n}$ as $a_n$ is nonnegative). Therefore for $n>N$, $a_n<a_n^p$. Then $\sum_{n=N+1}^\infty a_n^p<\sum_{n=N+1}^\infty a_n$ converges by comparison test and $\sum a_n^p$ converges.
    \item [14.12] Let $(a_n)_{n\in\mathbb{N}}$ be a sequence such that $\text{lim inf }\abs{a_n}=0$. Prove there is a subsequence $(a_{n_k})_{k\in\mathbb{N}}$ such that $\sum_{k=1}^\infty a_{n_k}$ converges.\\\textbf{Proof: } By Theorem 11.7, there exists a monotonic subsequence $(a_{n_k})$ of $(\abs{a_n})$ that converges to $\text{lim inf }\abs{a_n}=0$. Then take $\epsilon=\frac{1}{2}>0$, there exists an $N$ such that $n>N_1\implies\abs{a_{n_k}}<\epsilon=\frac{1}{2}$ by definition of convergence. We can find the next term by setting $\epsilon=\frac{1}{4}>0$, then there exists an $N_2$ such that $n>N_2\implies\abs{a_{n_k}}<\epsilon=\frac{1}{4}$. Following the same method, we can construct a series where each term $a_{n_k}$ is bounded by $\frac{1}{2^k}$. Then the series is bounded by $\sum\frac{1}{2^k}$ which is convergent, therefore it is also convergent by Comparison Test.
    \item [17.3] Prove the following functions are continuous:
          \begin{itemize}
              \item [(a)] $\text{log}_e(1+\text{cos}^4x)$:\\\textbf{Proof: } Since $\text{cos }x$ is given to be continuous, so is $\text{cos}^4x$ upon applying Theorem 17.4(ii) four times. Then $1+\text{cos}^4x$ by Theorem 17.4(i). Furthermore, since $\text{log}_ex$ is also continuous, the composite function $\text{log}_e(1+\text{cos}^4x)$ is continuous by Theorem 17.5.
              \item [(c)] $2^{x^2}$\\\textbf{Proof: } Since $x^2$ and $2^x$ are both continuous functions as given, their composite function $2^{x^2}$ is also continuous.
              \item [(f)] $x\,\text{sin}(\frac{1}{x})$ for $x\neq 0$\\\textbf{Proof: } $\frac{1}{x}=x^{-1}$ which is continuous as given; $\text{sin }x$ is also continuous therefore the composite $\text{sin}(\frac{1}{x})$ is continuous. In addition, $x=x^1$ which is continuous, then $x\,\text{sin}(\frac{1}{x})$ is continuous by Theorem 17.4(ii).
          \end{itemize}
    \item [17.5]
          \begin{itemize}
              \item [(a)] Prove that if $m\in\mathbb{N}$, then the function $f(x)=x^m$ is continuous on $\mathbb{R}$.\\\textbf{Proof: } Suppose $\text{lim }x_n=x_0$, then $\text{lim }f(x_n)=\text{lim }x^m=(\text{lim }x)^m=x_0^m=f(x_0)$. Therefore $f(x)=x^m$ is continuous by definition.
              \item [(b)] Prove every \textit{polynomial function} $p(x)=a_0+a_1x+\cdots+a_nx^n$ is continuous on $\mathbb{R}$.\\\textbf{Proof: } Since $x^m$ is continuous as shown above for $m\in\mathbb{N}$, $a_mx^m$ is also continuous by Theorem 17.3. Then $p(x)=a_0+a_1x+\cdots+a_nx^n$ is continuous upon applying Theorem 17.4(i) to all terms.
          \end{itemize}
    \item [17.6] A \textit{rational function} is a function $f$ of the form $p/q$ where $p$ and $q$ are polynomial functions. The domain of $f$ is $\{x\in\mathbb{R}:q(x)\neq 0\}$. Prove every rational function is continuous.\\\textbf{Proof: } Since all polynomial functions are continuous as shown in 17.5(b), all $p$ and $q$ are continuous. Then, with the domain constraint that $q(x)\neq 0$, $p/q$ is continuous by Theorem 17.4(iii).
    \item [17.9] Prove each of the following functions is continuous at $x_0$ by verifying the $\epsilon$-$\delta$ property of Theorem 17.2.
          \begin{itemize}
              \item [(b)] $f(x)=\sqrt{x},x_0=0$\\\textbf{Proof: } Let $\epsilon>0$ and $\delta=\epsilon^2$, then for $\abs{x-x_0}=\abs{x}<\delta=\epsilon^2$, we have $\abs{f(x)-f(x_0)}=\abs{\sqrt{x}}<\sqrt{\delta}=\epsilon$.
              \item [(d)] $g(x)=x^3,x_0$ arbitrary\\\textbf{Proof: } Let $x_0\in\mathbb{R}$ and $\epsilon>0$, then $\abs{f(x)-f(x_0)}=\abs{x^3-x_0^3}=\abs{(x-x_0)(x^2+xx_0+x_0^2)}=\abs{x-x_0}\cdot\abs{x^2+xx_0+x_0^2}$. To bound the second term, we can choose $\delta=1$, then $\abs{x-x_0}<1\implies\abs{x}<1+\abs{x_0}$; by substitution, $\abs{x^2+xx_0+x_0^2}\leq\abs{x}^2+\abs{x}\cdot\abs{x_0}+\abs{x_0}^2<(1+\abs{x_0})^2+\abs{x_0}(1+\abs{x_0})+\abs{x_0}^2=3\abs{x_0}^2+3\abs{x_0}+1$. Thus $\abs{x-x_0}\cdot\abs{x^2+xx_0+x_0^2}\leq\abs{x-x_0}(3\abs{x_0}^2+3\abs{x_0}+1)$. Then we can take $\delta=\text{min}\left\{1,\frac{\epsilon}{3\abs{x_0}^2+3\abs{x_0}+1}\right\}$ and by construction $\abs{x-x_0}<\delta\implies\abs{f(x)-f(x_0)}<\epsilon$, therefore $g(x)$ is continuous.
          \end{itemize}
    \item [17.10] Prove the following functions are discontinuous at the indicated points.
          \begin{itemize}
              \item [(b)] $g(x)=\text{sin}(\frac{1}{x})$ for $x\neq 0$ and $g(0)=0,x_0=0$\\\textbf{Proof: } Assume that $g(x)$ is continuous, then by definition of continuity, $\text{lim }g(x)=\text{lim }\text{sin}(\frac{1}{x})=g(x_0)=0$. However, $\text{sin}(t)$ is not convergent and therefore does not have a limit. Then our assumption is false and $g(x)$ is discontinuous.
          \end{itemize}
    \item [17.12]
          \begin{itemize}
              \item [(a)] Let $f$ be a continuous real-valued function with domain $(a,b)$. Show that if $f(r)=0$ for each rational number $r$ in $(a,b)$, then $f(x)=0$ for all $x\in(a,b)$.\\\textbf{Proof: } Since $(a,b)\in\mathbb{R}$ and $\mathbb{Q}$ is dense in $\mathbb{R}$, there exists a rational sequence $r_n$ such that $r_n\rightarrow x$, then $f(r_n)=0$ since $f(r)=0$ for $r\in(a,b)$. Since $f$ is continuous, by definition of continuity, we have $f(x)=\text{lim}_nf(r_n)=0$.
              \item [(b)] Let $f$ and $g$ be continuous real-valued functions on $(a,b)$ such that $f(r)=g(r)$ for each rational number $r$ in $(a,b)$. Prove $f(x)=g(x)$ for all $x\in(a,b)$.\\\textbf{Proof: } Let $h=f-g$, then $h$ is continuous by Theorems 17.3 and 17.4(i). Since $f(r)=g(r)$, we have $h(r)=0$ for $r\in(a,b)$. Then by part (a), for all $x\in(a,b)$, $h(x)=0\implies f(x)-g(x)=0\implies f(x)=g(x)$.
          \end{itemize}
    \item [18.5]
          \begin{itemize}
              \item [(a)] Let $f$ and $g$ be continuous functions on $[a,b]$ such that $f(a)\geq g(a)$ and $f(b)\leq g(b)$. Prove $f(x_0)=g(x_0)$ for at least one $x_0$ in $[a,b]$.\\\textbf{Proof: } Since $f$ and $g$ are both continuous, so is $h=f+(-1)g=f-g$ by Theorems 17.3 and 17.4(i). Then, since $f(a)\geq g(a)$, $h(a)\geq 0$. Similarly, since $f(b)\leq g(b)$, $h(b)\leq 0$. Then $h$ must have a root on the interval $[a,b]$ by Intermediate Value Theorem, i.e. $h(x_0)=0\implies f(x_0)-g(x_0)=0\implies f(x_0)=g(x_0)$.
          \end{itemize}
    \item [18.6] Prove $x=\text{cos}\,x$ for some $x$ in $(0,\frac{\pi}{2})$.\\\textbf{Proof: } Let $f(x)=\text{cos}\,x$ and $g(x)=x$. Since $f(0)=\text{cos}\,0=1$ and $g(0)=0$, $f(0)\geq g(0)$. Similarly, since $f(\frac{\pi}{2})=0$ and $g(\frac{\pi}{2})$, we have $f(\frac{\pi}{2})\leq g(\frac{\pi}{2})$. Then there must exist at least one $x_0$ in $[0,\frac{\pi}{2}]$ such that $f(x_0)=g(x_0)$ as shown in Exercise 18.5(a).
    \item [18.9] Prove that a polynomial function $f$ of odd degree has at least one real root.\\\textbf{Proof: } Since $f$ is a polynomial of odd degree, we have $f(x)=a_0+a_1x+\ldots+a_nx^n$ where $n$ is odd and $a_n\neq 0$. Suppose $a_n>0$ and since $x^n$ diverges, $\text{lim}_{x\rightarrow-\infty}f(x)=-\infty$ and $\text{lim}_{x\rightarrow+\infty}f(x)=\infty$. Then $f(x)$ must have a root on $(-\infty,+\infty)$ by Intermediate Value Theorem. Similarly, if $a_n<0$, we have $\text{lim}_{x\rightarrow-\infty}f(x)=+\infty$ and $\text{lim}_{x\rightarrow+\infty}f(x)=-\infty$ and again $f(x)$ must have a root on $(-\infty,+\infty)$ by Intermediate Value Theorem.
    \item [P1] Let $f,g,h$ be real-valued functions defined on subsets of $\mathbb{R}$. Suppose for all $x$, $f(x)\leq g(x)\leq h(x)$, $f(x_0)=h(x_0)$, and $f$ and $h$ are continuous at $x_0$. Prove that $g$ is continuous at $x_0$.\\\textbf{Proof: } Since $f$ is continuous at $x_0$, then by definition of continuity, we have $\text{lim}_nf(x_n)=x_0$ for all sequences $(x_n)$ converging to $x_0$. Similarly, we also have $\text{lim}_nf(x_n)=x_0$ for all sequences $(x_n)\rightarrow x_0$. Then, since $f(x)\leq g(x)\leq h(x)$ for all $x$, applying the $f,g,h$ to $x_n$ gives us $f(x_n)\leq g(x_n)\leq h(x_n)$. Therefore $\text{lim}_nf(x_n)=\text{lim}_ng(x_n)=\text{lim}_hf(x_n)=x_0$ by squeeze lemma, and $g$ is continuous at $x_0$ by definition of continuity.
\end{itemize}

\end{document}