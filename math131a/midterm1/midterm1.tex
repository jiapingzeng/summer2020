\documentclass{article}
\usepackage[margin=1in]{geometry}
\usepackage{setspace}
\usepackage{amsmath}
\usepackage{amssymb}
\usepackage{physics}

\title{Math 131A Midterm 1}
\author{Jiaping Zeng}
\date{8/18/2020}

\begin{document}
\setstretch{1.5}
\begin{itemize}
    \item [1.]
          \begin{itemize}
              \item [(a)] Let $\epsilon>0$. Let $x$ and $y$ be real numbers such that $\abs{x-1}<\frac{\epsilon}{2}$ and $\abs{y-2}<\frac{\epsilon}{2}$. Prove that $\abs{x-y}<\epsilon+1$.\\
                    \textbf{Proof:} By expanding the absolute value, we have $\abs{x-1}<\frac{\epsilon}{2}\implies -\frac{\epsilon}{2}<x-1<\frac{\epsilon}{2}\implies 1-\frac{\epsilon}{2}<x<1+\frac{\epsilon}{2}$. Similarly, we also have $2-\frac{\epsilon}{2}<y<2+\frac{\epsilon}{2}$, which implies $-2+\frac{\epsilon}{2}>-y>-2-\frac{\epsilon}{2}$. Then, we can add the two inequalities as follows: $(1-\frac{\epsilon}{2})+(-2-\frac{\epsilon}{2})<x+(-y)<(1+\frac{\epsilon}{2})+(-2+\frac{\epsilon}{2})$, which simplies to $-\epsilon-1<x-y<\epsilon-1$. Since $\epsilon+1>\epsilon-1$, $-\epsilon-1<x-y<\epsilon+1$, therefore $\abs{x-y}<\epsilon+1$.
              \item [(b)] Let $x$ be a real number such that $\abs{x}\leq\frac{3}{4}$ and $\abs{x-1}\leq\frac{3}{4}$. Prove that $\abs{x-\frac{1}{2}}\leq\frac{1}{4}$.\\
                    \textbf{Proof:}  Again by expanding the absolute values, we have $-\frac{3}{4}\leq x\leq\frac{3}{4}$ and $1-\frac{3}{4}\leq x\leq 1+\frac{3}{4}\implies \frac{1}{4}\leq x\leq\frac{7}{4}$. By combining the two inequalities and taking the tighter bounds, we have $\frac{1}{4}\leq x\leq\frac{3}{4}$. Then, we can subtract $\frac{1}{2}$ from the inequality to obtain $-\frac{1}{4}\leq x-\frac{1}{2}\leq\frac{1}{4}$; therefore $\abs{x-\frac{1}{2}}\leq\frac{1}{4}$.
          \end{itemize}
\end{itemize}
\newpage
\begin{itemize}
    \item [2.] Let $S\subseteq\mathbb{R}$ be nonempty and bounded above. Let $T=\{2x+1:x\in S\}$. Prove that $\text{sup }T$ exists and that \[\text{sup }T=2\cdot\text{sup }S+1.\]\\
          \textbf{Proof:} Since $\text{sup }S\geq x$ for all $x\in S$ by definition of supremum, $2\cdot\text{sup }S+1\geq 2x+1\in T$. Therefore $2\cdot\text{sup }S+1$ is an upper bound of $T$, so $T$ is bounded above and $\text{sup }T$ exists. Then, since $2\cdot\text{sup }S+1$ is an upper bound and $\text{sup }T$ exists, we have $\text{sup }T\leq 2\cdot\text{sup }S+1$.\\
          Now suppose $\text{sup }T<2\cdot\text{sup }S+1$; then there must exist an $r\in\mathbb{Q}$ such that $\text{sup }T<r<2\cdot\text{sup }S+1$ since $\mathbb{Q}$ is dense in $\mathbb{R}$. Then, since $\text{sup }T>2x+1$ for all $x\in S$, we also have $2\cdot\text{sup }S+1>r>\text{sup }T>2x+1$. By subtracting $1$ and dividing by $2$, the previous expression gives us $\text{sup }S>\frac{r-1}{2}>x$ for $x\in S$, which implies that $\frac{r-1}{2}$ is an upper bound of $S$ while being strictly less than the supremum of the set. This cannot happen by definition of supremum, therefore $\text{sup }T\nless 2\cdot\text{sup }S+1$.\\
          Combining $\text{sup }T\leq 2\cdot\text{sup }S+1$ and $\text{sup }T\nless 2\cdot\text{sup }S+1$ gives us $\text{sup }T=2\cdot\text{sup }S+1$.
\end{itemize}
\newpage
\begin{itemize}
    \item [3.] Let $(s_n)$ and $(t_n)$ be convergent sequences of real numbers. Suppose $\text{lim }s_n=\text{lim }t_n+1$. Prove there exists $N$ such that $s_n-t_n>\frac{1}{2}$ for all $n>N$.\\
          \textbf{Proof:} Let $s=\text{lim }s_n$ and $t=\text{lim }t_n$, also let $\frac{1}{4}\geq\epsilon>0$. By definition of convergence, there exists an $N_s$ such that $n>N_s\implies\abs{s_n-s}<\epsilon\implies s-\epsilon<s_n<s+\epsilon$; similarly, there exists an $N_t$ such that $n>N_t\implies\abs{t_n-t}<\epsilon\implies t-\epsilon<t_n<t+\epsilon$. Then $s-\epsilon-t-\epsilon<s_n-t_n<s+\epsilon-t+\epsilon$ which simplifies to $s-t-2\epsilon<s_n-t_n<s-t+2\epsilon$. Since $s=t+1$, we have $1-2\epsilon<s_n-t_n$ by substitution. Since we selected $\epsilon\leq\frac{1}{4}$, $N=\text{max}\{N_s,N_t\}$ will guarantee that $n>N\implies s_n-t_n>\frac{1}{2}$.
\end{itemize} 
\newpage
\begin{itemize}
    \item [4.] Let $(s_n)$ be a sequence defined by \[s_n=\dfrac{2n^2+3}{n(5n-1)}, n=1,2,\ldots\] Prove that $(s_n)$ is convergent and find the value of $\text{lim }s_n$.\\
          \textbf{Proof:}\\$\text{lim }s_n=\text{lim }\dfrac{2n^2+3}{n(5n+1)}$\\$=\text{lim }\dfrac{2n^2+3}{5n^2+n}$ (Expand denominator)\\$=\text{lim }\dfrac{2+\frac{3}{n^2}}{5+\frac{1}{n}}$ (Multiply by $\dfrac{\frac{1}{n^2}}{\frac{1}{n^2}}$)\\$=\dfrac{\text{lim }(2+\frac{3}{n^2})}{\text{lim }(5+\frac{1}{n})}$ (Lemma 22)\\$=\dfrac{\text{lim }2+\text{lim }\frac{3}{n^2}}{\text{lim }5+\text{lim }\frac{1}{n}}$ (Sum limit law)\\$=\dfrac{2+3\cdot\text{lim }\frac{1}{n^2}}{5+\text{lim }\frac{1}{n}}$ (Theorem 19)\\$=\dfrac{2+3\cdot 0}{5+0}$ ($\text{lim }\frac{1}{n^2}=\text{lim }\frac{1}{n}=0$ shown in class)\\$=\dfrac{2}{5}$\\Therefore $(s_n)$ converges to $\dfrac{2}{5}$.

\end{itemize}
\newpage
\begin{itemize}
    \item [5.] Let $(s_n)$ and $(t_n)$ be sequences of real numbers. Suppose $(s_n)$ diverges to $+\infty$ and that the set \[\{n\in\mathbb{N}:t_n<s_n\}\] is finite. Prove that $(t_n)$ diverges to $+\infty$.\\
          \textbf{Proof:} Since $(s_n)$ diverges to $+\infty$, there exists an $N_s$ such that $n>N_s\implies s_n>M$ for $M>0$. In addition, since the set $\{n\in\mathbb{N}:t_n<s_n\}$ is finite, there exists an $N_0$ such that $n>N_0\implies t_n\geq s_n$. Then we can select $N=\text{max}\{N_s,N_0\}$ and we have $n>N\implies t_n\geq s_n>M$. Therefore $n>N\implies t_n>M$ and $(t_n)$ diverges to $+\infty$ by definition of divergence.
\end{itemize}
\end{document}