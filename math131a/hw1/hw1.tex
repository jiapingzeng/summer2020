\documentclass{article}
\usepackage[margin=1in]{geometry}
\usepackage{enumitem}
\usepackage{setspace}
\usepackage{amsmath}
\usepackage{amssymb}

\title{Math 131A Homework 1}
\date{8/3/2020}
\author{Jiaping Zeng}

\begin{document}
\setstretch{1.35}
\maketitle

\begin{itemize}
    \item [1.1] Prove $1^2+2^2+...+n^2=\dfrac{1}{6}n(n+1)(2n+1)$ for all positive integers n.\\\textbf{Answer: } By induction. \\
    Base case ($n=1$): $1=\dfrac{1}{6}(1+1)(2+1)\implies1=1$ which is true. \\
    Inductive step: Assume $1^2+2^2+...+n^2=\dfrac{1}{6}n(n+1)(2n+1)$ is true, we want to show that $1^2+2^2+...+n^2+(n+1)^2=\dfrac{1}{6}(n+1)(n+2)(2n+3)$ is true. We can do so by adding $(n+1)^2$ to both sides of the equation as follows: 
    \[1^2+2^2+...+n^2+(n+1)^2=\dfrac{1}{6}n(n+1)(2n+1)+(n+1)^2\]
    Factoring out $(n+1)$ from both terms on the right hand side:
    \[1^2+2^2+...+n^2+(n+1)^2=(n+1)(\dfrac{1}{6}n(2n+1)+(n+1))\]
    \[\implies1^2+2^2+...+n^2+(n+1)^2=(n+1)(\dfrac{1}{6}n(2n+1)+(n+1))\]
    \item [1.9]
          \begin{itemize}
              \item [(a)] Decide for which integers the inequality $2^n>n^2$ is true.
              \item [(b)] Prove your claim in (a) by mathematical induction.
          \end{itemize}
    \item [1.11] For each $n\in\mathbb{N}$, let $P_n$ denote the assertion "$n^2+5n+1$ is an even integer."
          \begin{itemize}
              \item [(a)] Prove $P_{n+1}$ is true whenever $P_n$ is true.
              \item [(b)] For which $n$ is $P_n$ actually true? What is the moral of this exercise?
          \end{itemize}
    \item [3.1]
    \item [3.6]
    \item [3.7]
    \item [3.8]
    \item [4.6]
    \item [4.7]
    \item [4.8]
    \item [4.14]
    \item [4.15]
    \item [4.16]
    \item [P1] Write down the converse and the contrapositive of the following statement regarding a real number $x$: \[\text{If } x>0, \text{then } x^2-x>0.\] Then determine which (if any) of the three statements are true for all real numbers $x$.\\
          \textbf{Answer: } Converse: if $x^2-x>0$, then $x>0$, which is false by counterexample $x=-1$. Contrapositive: if $x^2-x\leq0$, then $x\leq0$, which is false by counterexample $x=1$.
    \item [P2] Prove that $\sqrt{3}$ is not rational.\\
          \textbf{Answer: } By contradiction. Suppose $\sqrt{3}$ is rational, then by definition of rational numbers, there must exist $p,q\in\mathbb{Z}$ such that $\dfrac{p}{q}=\sqrt{3}$. By rearranging we have $p=\sqrt{3}q$; however, since $\sqrt{3}$ is not an integer, $\sqrt{3}q$ is also not an integer. Then by extension $p$ is not an integer, which contradicts our initial assumption. Therefore $\sqrt{3}$ is not rational.
\end{itemize}

\end{document}