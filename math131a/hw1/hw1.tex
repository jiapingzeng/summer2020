\documentclass{article}
\usepackage[margin=1in]{geometry}
\usepackage{enumitem}
\usepackage{setspace}
\usepackage{amsmath}
\usepackage{amssymb}

\title{Math 131A Homework 1}
\date{8/3/2020}
\author{Jiaping Zeng}

\begin{document}
\setstretch{1.35}
\maketitle

\begin{itemize}
      \item [1.1] Prove $1^2+2^2+...+n^2=\dfrac{1}{6}n(n+1)(2n+1)$ for all positive integers $n$.\\\textbf{Answer: } By induction. \\
            Base case ($n=1$): $1=\dfrac{1}{6}(1+1)(2+1)\implies1=1$ which is true. \\
            Inductive step: Assume $1^2+2^2+...+n^2=\dfrac{1}{6}n(n+1)(2n+1)$ is true, we want to show that $1^2+2^2+...+n^2+(n+1)^2=\dfrac{1}{6}(n+1)(n+2)(2n+3)$ is true. We can do so by adding $(n+1)^2$ to both sides of the equation as follows:
            \[1^2+2^2+...+n^2+(n+1)^2=\dfrac{1}{6}n(n+1)(2n+1)+(n+1)^2\]
            Expanding the right hand side results in the following:
            \[1^2+2^2+...+n^2+(n+1)^2=\dfrac{1}{3}n^3+\dfrac{3}{2}n^2+\dfrac{13}{6}n+1\]
            Which indeed factors into
            \[\implies1^2+2^2+...+n^2+(n+1)^2=\dfrac{1}{6}(n+1)(n+2)(2n+3).\]
            Therefore $1^2+2^2+...+n^2=\dfrac{1}{6}n(n+1)(2n+1)$ for all positive integers $n$ by mathematical induction.
      \item [1.9]
            \begin{itemize}
                  \item [(a)] Decide for which integers the inequality $2^n>n^2$ is true.\\
                        \textbf{Answer: } $2^n>n^2, n\in\mathbb{Z}$ is true for $n=0,1$ and $n>4$.
                  \item [(b)] Prove your claim in (a) by mathematical induction.\\
                        \textbf{Answer: } We will first show $n=0$ and $n=1$ case-by-case, then $n>4$ by induction.\\
                        $n=0$: $2^n>n^2\implies 1>0$ which is true\\
                        $n=1$: $2^n>n^2\implies 2>1$ which is true\\
                        $n>4$: By induction as follows.\\
                        Base case: ($n=5$) $2^n>n^2\implies 32>25$ which is true\\
                        Indcutive step: Assume $2^n>n^2$ is true, we want to show that $2^{n+1}>(n+1)^2$ is also true. We can start by multiplying $2$ to both sides of the inequality: $2*2^n>2*n^2\implies 2^{n+1}>2n^2$. Then, if we could show that $2n^2>(n+1)^2$ for $n\geq 5$, $2^{n+1}>(n+1)^2$ would be also true. We will do so using another proof by induction: \begin{itemize}
                              \item [] Base case: $(n=5)$: $2n^2>(n+1)^2\implies 50>36$ which is true
                              \item [] Inductive step: Assume $2n^2>(n+1)^2$ is true, we want to show that $2(n+1)^2>(n+2)^2$. By expanding the right hand side of the assumption, we have $2n^2>n^2+2n+1\implies n^2>2n+1$. $2(n+1)^2=2n^2+4n+2, (n+2)^2=n^2+4n+4$
                        \end{itemize}

            \end{itemize}
      \item [1.11] For each $n\in\mathbb{N}$, let $P_n$ denote the assertion "$n^2+5n+1$ is an even integer."
            \begin{itemize}
                  \item [(a)] Prove $P_{n+1}$ is true whenever $P_n$ is true.\\
                        \textbf{Answer: } Using the definition of $P_n$, $P_{n+1}$ corresponds to the expression $(n+1)^2+5(n+1)+1$. To show that it is even, we can first expand it into $(n+1)^2+5(n+1)+1=n^2+7n+7$. Then, $P_{n+1}-P_n=n^2+7n+7-n^2-5n-1=2n+6$ which is always even for $n\in\mathbb{N}$. Therefore, if $P_n$ is even, then $P_n+(P_{n+1}-P_n)=P_{n+1}$ must also be even.
                  \item [(b)] For which $n$ is $P_n$ actually true? What is the moral of this exercise?\\
                        \textbf{Answer: } For $n^2+5n+1$ to be even, $n^2+5n=n(n+5)$ must be odd. However, this is not possible for $n\in\mathbb{N}$. On one hand, if $n$ is even, $n(n+5)$ would have a factor of $2$ from $n$ and would therefore be even; on the other hand, if $n$ is odd, then $(n+5)$ would be even and $n(n+5)$ as well. The moral of this exercise is that finding a true base case is important in mathematical induction.
            \end{itemize}
      \item [3.1]
            \begin{itemize}
                  \item [(a)] Which of the properties A1-A4, M1-M4, DL, O1-O5 fail for $\mathbb{N}$?\\
                        \textbf{Answer: } A4 fails since there are no negative numbers in $\mathbb{N}$; M4 fails since there are no fractions.
                  \item [(b)] Which of these properties fail for $\mathbb{Z}$?\\
                        \textbf{Answer: } M4 fails since there are no fractions in $\mathbb{Z}$.
            \end{itemize}
      \item [3.6]
            \begin{itemize}
                  \item [(a)] Prove $|a+b+c|\leq|a|+|b|+|c|$ for all $a,b,c\in\mathbb{R}$.\\
                        \textbf{Answer: } Let $m=a+b$, then since $|a+b|\leq |a|+|b|$ (triangle inequality), $|m|\leq |a|+|b|$. Then we can add $|c|$ to both sides of the inequality, resulting in $|m|+|c|\leq |a|+|b|+|c|$. Since $|m+c|\leq|m|+|c|$, we have $|m+c|\leq|m|+|c|\leq|a|+|b|+|c|\implies|m+c|\leq|a|+|b|+|c|$. Then by substituting $m=a+b$ we have $|a+b+c|\leq|a|+|b|+|c|$.
                  \item [(b)] Use induction to prove \[|a_1+a_2+\ldots+a_n|\leq|a_1|+|a_2|+\ldots+|a_n|\] for $n$ numbers $a_1,a_2,...,a_n$.\\
                        \textbf{Answer: }\\
                        Base case: $(n=1)$: $|a_1|\leq|a_1|$ which is true.\\
                        Inductive step: Assume the statement holds for $n$ numbers $a_1,a_2,\ldots,a_n$. Let $m=a_1+a_2+\ldots+a_n$, then we have $|m|\leq|a_1|+|a_2|+\ldots+|a_n|$. Adding $|a_{n+1}|$ to both sides of the inequality gives us $|m|+|a_{n+1}|\leq|a_1|+|a_2|+\ldots+|a_n|+|a_{n+1}|$. Then using $|m+a_{n+1}|\leq|m|+|a_{n+1}|$ (similar to previous part), we have $|m+a_{n+1}|\leq|a_1|+|a_2|+\ldots+|a_n|+|a_{n+1}|$. Then by substitution, $|a_1+a_2+\ldots+a_n+a_{n+1}|\leq|a_1|+|a_2|+\ldots+|a_n|+|a_{n+1}|$.\\
                        Therefore $|a_1+a_2+\ldots+a_n|\leq|a_1|+|a_2|+\ldots+|a_n|$ by mathematical induction.
            \end{itemize}
      \item [3.7]
            \begin{itemize}
                  \item [(a)] Show $|b|<a$ if and only if $-a<b<a$.\\
                        \textbf{Answer: } 
                        \begin{itemize}
                              \item [$\Rightarrow$:] By cases. \\Case 1 ($b\geq 0$): we have $|b|=b$ by definition of aboslute value. Then, $|b|<a\implies b<a$. \\Case 2 ($b\leq 0$): $|b|=-b$, then $|b|<a\implies -b<a\implies b>-a$. \\Therefore $-a<b<a$.
                              \item [$\Leftarrow$:] Also by cases. \\Case 1 ($b\geq 0$): $b=|b|$ by definition; then $-a<|b|<a$. \\Case 2 ($b\leq 0)$: $-a<b<a\implies a>-b>-a$ (multiply by $-1$). Since $-b=|b|$, $a>-b>-a\implies a>|b|>-a\implies -a<|b|<a$. \\Therefore $|b|<a$.
                        \end{itemize}
                  \item [(b)] Show $|a-b|<c$ if and only if $b-c<a<b+c$.
                  \item [(c)] Show $|a-b|\leq c$ if and only if $b-c\leq a\leq b+c$.
            \end{itemize}
      \item [3.8] Let $a,b\in\mathbb{R}$. Show if $a\leq b_1$ for every $b_1>b$, then $a\leq b$.
      \item [4.6]
      \item [4.7]
      \item [4.8]
      \item [4.14]
      \item [4.15]
      \item [4.16]
      \item [P1] Write down the converse and the contrapositive of the following statement regarding a real number $x$: \[\text{If } x>0, \text{then } x^2-x>0.\] Then determine which (if any) of the three statements are true for all real numbers $x$.\\
            \textbf{Answer: } Converse: if $x^2-x>0$, then $x>0$, which is false by counterexample $x=-1$. Contrapositive: if $x^2-x\leq0$, then $x\leq0$, which is false by counterexample $x=1$.
      \item [P2] Prove that $\sqrt{3}$ is not rational.\\
            \textbf{Answer: } By contradiction. Suppose $\sqrt{3}$ is rational, then by definition of rational numbers, there must exist $p,q\in\mathbb{Z}$ such that $\dfrac{p}{q}=\sqrt{3}$, where $p,q$ have no common factors upon simplying. Then, we also have $\dfrac{p^2}{q^2}=3\implies p^2=3q^2$. Since $p,q\in\mathbb{Z}$ and by extension $p^2,q^2\in\mathbb{Z}$, $p^2\mid 3$. Additionally, $p\mid 3$ because $\sqrt{3}\notin\mathbb{Z}$. Then, $p^2\mid 9$. By substituting $p^2=3q^2$, we now have $3q^2\mid 9\implies q^2\mid 3$, implying $q\mid 3$ by previous logic. Then $p,q$ have common factor $3$ whcih contradicts with our initial assumption. Therefore $\sqrt{3}$ is not rational.
\end{itemize}

\end{document}