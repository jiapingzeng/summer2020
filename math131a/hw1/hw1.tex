\documentclass{article}
\usepackage[margin=1in]{geometry}
\usepackage{enumitem}
\usepackage{setspace}
\usepackage{amsmath}
\usepackage{amssymb}

\title{Math 131A Homework 1}
\date{8/3/2020}
\author{Jiaping Zeng}

\begin{document}
\setstretch{1.35}
\maketitle

\begin{itemize}
    \item [1.1] Prove $1^2+2^2+...+n^2=\dfrac{1}{6}n(n+1)(2n+1)$ for all positive integers $n$.\\\textbf{Answer: } By induction. \\
          Base case ($n=1$): $1=\dfrac{1}{6}(1+1)(2+1)\implies1=1$ which is true. \\
          Inductive step: Assume $1^2+2^2+...+n^2=\dfrac{1}{6}n(n+1)(2n+1)$ is true, we want to show that $1^2+2^2+...+n^2+(n+1)^2=\dfrac{1}{6}(n+1)(n+2)(2n+3)$ is true. We can do so by adding $(n+1)^2$ to both sides of the equation as follows:
          \[1^2+2^2+...+n^2+(n+1)^2=\dfrac{1}{6}n(n+1)(2n+1)+(n+1)^2\]
          Expanding the right hand side results in the following:
          \[1^2+2^2+...+n^2+(n+1)^2=\dfrac{1}{3}n^3+\dfrac{3}{2}n^2+\dfrac{13}{6}n+1\]
          Which indeed factors into
          \[\implies1^2+2^2+...+n^2+(n+1)^2=\dfrac{1}{6}(n+1)(n+2)(2n+3).\]
          Therefore $1^2+2^2+...+n^2=\dfrac{1}{6}n(n+1)(2n+1)$ for all positive integers $n$ by mathematical induction.
    \item [1.9]
          \begin{itemize}
              \item [(a)] Decide for which integers the inequality $2^n>n^2$ is true.\\
                    \textbf{Answer: } $2^n>n^2, n\in\mathbb{Z}$ is true for $n=0,1$ and $n\geq 5$.
              \item [(b)] Prove your claim in (a) by mathematical induction.\\
                    \textbf{Answer: } We will first show $n=0$ and $n=1$ case-by-case, then $n\geq 5$ by induction.\\
                    $n=0$: $2^n>n^2\implies 1>0$ which is true\\
                    $n=1$: $2^n>n^2\implies 2>1$ which is true\\
                    $n\geq 5$: By induction as follows.\\
                    Base case: ($n=5$) $2^n>n^2\implies 32>25$ which is true\\
                    Indcutive step: Assume $2^n>n^2$ is true, we want to show that $2^{n+1}>(n+1)^2$ is also true. We can start by multiplying $2$ to both sides of the inequality: $2*2^n>2*n^2\implies 2^{n+1}>2n^2$. Then, if we could show that $2n^2>(n+1)^2$ for $n\geq 5$, $2^{n+1}>(n+1)^2$ would be also true. By expanding the right hand side and subtracting $n^2$ from both sides, $2n^2>(n+1)^2$ simplifies to $n^2>2n+1$. We will prove this inequality for $n>5$ using another proof by induction: \begin{itemize}
                        \item [] Base case: $(n=5)$: $n^2>2n+1\implies 25>11$ which is true
                        \item [] Inductive step: Assume $n^2>2n+1$ is true, we want to show that $(n+1)^2>2n+3$. By expanding the left hand side and canceling terms, we have $n^2>2$ which is clearly true for $n>5$.
                    \end{itemize}
                    Therefore $n^2>2n+1$, and by extension $2^{n+1}>(n+1)^2$, so $2^n>n^2$ is true for $n=0,1$ and $n\geq 5$ by mathematical induction.
          \end{itemize}
    \item [1.11] For each $n\in\mathbb{N}$, let $P_n$ denote the assertion ``$n^2+5n+1$ is an even integer.''
          \begin{itemize}
              \item [(a)] Prove $P_{n+1}$ is true whenever $P_n$ is true.\\
                    \textbf{Answer: } Using the definition of $P_n$, $P_{n+1}$ corresponds to the expression $(n+1)^2+5(n+1)+1$. To show that it is even, we can first expand it into $(n+1)^2+5(n+1)+1=n^2+7n+7$. Then, $P_{n+1}-P_n=n^2+7n+7-n^2-5n-1=2n+6$ which is always even for $n\in\mathbb{N}$. Therefore, if $P_n$ is even, then $P_n+(P_{n+1}-P_n)=P_{n+1}$ must also be even.
              \item [(b)] For which $n$ is $P_n$ actually true? What is the moral of this exercise?\\
                    \textbf{Answer: } For $n^2+5n+1$ to be even, $n^2+5n=n(n+5)$ must be odd. However, this is not possible for $n\in\mathbb{N}$. On one hand, if $n$ is even, $n(n+5)$ would have a factor of $2$ from $n$ and would therefore be even; on the other hand, if $n$ is odd, then $(n+5)$ would be even and $n(n+5)$ as well. The moral of this exercise is that finding a true base case is important in mathematical induction.
          \end{itemize}
    \item [3.1]
          \begin{itemize}
              \item [(a)] Which of the properties A1-A4, M1-M4, DL, O1-O5 fail for $\mathbb{N}$?\\
                    \textbf{Answer: } A3 fails since there is no $0\in\mathbb{N}$; A4 fails since there are no negative numbers in $\mathbb{N}$; M4 fails since there are no fractions.
              \item [(b)] Which of these properties fail for $\mathbb{Z}$?\\
                    \textbf{Answer: } M4 fails since there are no fractions in $\mathbb{Z}$.
          \end{itemize}
    \item [3.6]
          \begin{itemize}
              \item [(a)] Prove $|a+b+c|\leq|a|+|b|+|c|$ for all $a,b,c\in\mathbb{R}$.\\
                    \textbf{Answer: } Let $m=a+b$, then since $|a+b|\leq |a|+|b|$ (triangle inequality), $|m|\leq |a|+|b|$. Then we can add $|c|$ to both sides of the inequality, resulting in $|m|+|c|\leq |a|+|b|+|c|$. Since $|m+c|\leq|m|+|c|$, we have $|m+c|\leq|m|+|c|\leq|a|+|b|+|c|\implies|m+c|\leq|a|+|b|+|c|$. Then by substituting $m=a+b$ we have $|a+b+c|\leq|a|+|b|+|c|$.
              \item [(b)] Use induction to prove \[|a_1+a_2+\ldots+a_n|\leq|a_1|+|a_2|+\ldots+|a_n|\] for $n$ numbers $a_1,a_2,...,a_n$.\\
                    \textbf{Answer: }\\
                    Base case: $(n=1)$: $|a_1|\leq|a_1|$ which is true.\\
                    Inductive step: Assume the statement holds for $n$ numbers $a_1,a_2,\ldots,a_n$. Let $m=a_1+a_2+\ldots+a_n$, then we have $|m|\leq|a_1|+|a_2|+\ldots+|a_n|$. Adding $|a_{n+1}|$ to both sides of the inequality gives us $|m|+|a_{n+1}|\leq|a_1|+|a_2|+\ldots+|a_n|+|a_{n+1}|$. Then using $|m+a_{n+1}|\leq|m|+|a_{n+1}|$ (similar to previous part), we have $|m+a_{n+1}|\leq|a_1|+|a_2|+\ldots+|a_n|+|a_{n+1}|$. Then by substitution, $|a_1+a_2+\ldots+a_n+a_{n+1}|\leq|a_1|+|a_2|+\ldots+|a_n|+|a_{n+1}|$.\\
                    Therefore $|a_1+a_2+\ldots+a_n|\leq|a_1|+|a_2|+\ldots+|a_n|$ by mathematical induction.
          \end{itemize}
    \item [3.7]
          \begin{itemize}
              \item [(a)] Show $|b|<a$ if and only if $-a<b<a$.\\
                    \textbf{Answer: }
                    \begin{itemize}
                        \item [$\Rightarrow$:] Case $b\geq 0$: we have $|b|=b$ by definition of aboslute value. Then, $|b|<a\implies b<a$. \\Case $b<0$: $|b|=-b$, then $|b|<a\implies -b<a\implies b>-a$. \\Therefore $-a<b<a$.
                        \item [$\Leftarrow$:] Case $b\geq 0$: $b=|b|$ by definition; then $-a<|b|<a$. \\Case $b<0$: $-a<b<a\implies a>-b>-a$ (multiply by $-1$). Since $-b=|b|$, $a>-b>-a\implies a>|b|>-a\implies -a<|b|<a$. \\Therefore $|b|<a$.
                    \end{itemize}
              \item [(b)] Show $|a-b|<c$ if and only if $b-c<a<b+c$.\\
                    \textbf{Answer: } $b-c<a<b+c\implies -c<a-b<c$ (subtract b). Then, let $m=a-b$. By substitution, the original statement is equivalent to ``$|m|<c$ if and only if $-c<m<c$'', which is true by part (a).
              \item [(c)] Show $|a-b|\leq c$ if and only if $b-c\leq a\leq b+c$.\\
                    \textbf{Answer: } Let $m=a-b$, then by substitution we would need to prove that $|m|\leq c$ if and only if $-c\leq m\leq c$ (subtract b).
                    \begin{itemize}
                        \item [$\Rightarrow$:]
                              Case $m\geq 0$: Since $|m|=m$ in this case, we have $m=|m|\leq c\implies m\leq c$.\\
                              Case $m<0$: $|m|=-m$, then similarly we have $-m\leq c \implies m\geq -c$.\\
                              Therefore $-c\leq m\leq c$.
                        \item [$\Leftarrow$:]
                              Case $m\geq 0$: Since $m=|m|$ in this case, we have $-c\leq |m|\leq c\implies |m|\leq c$. \\
                              Case $m<0$: Since $-m=|m|$ here, we have $-c\leq -|m|\leq c\implies c\geq |m|\geq -c$ (multiply by -1). Then $|m|\leq c$.\\
                              Therefore $|m|\leq c$ in both cases.
                    \end{itemize}
          \end{itemize}
    \item [3.8] Let $a,b\in\mathbb{R}$. Show if $a\leq b_1$ for every $b_1>b$, then $a\leq b$.\\
          \textbf{Answer: } By contradiction. Suppose $a\leq b_1$ for every $b_1>b$ and $a>b$. Since $a>b$, there must exist a $b_1\in\mathbb{Q}(\in\mathbb{R})$ where $b<b_1<a$ (Since $\mathbb{Q}$ is dense in $\mathbb{R}$). However, that would imply that there exists a $b_1$ where $a>b_1$ which contradicts our assumption. Therefore if $a\leq b_1$ for every $b_1>b$, then $a\leq b$.
    \item [4.6] Let $S$ be a nonempty bounded subset of $\mathbb{R}$.
          \begin{itemize}
              \item [(a)] Prove inf $S\leq$ sup $S$.\\
                    \textbf{Answer: } By definition, sup $S\geq s$ and inf $S\leq s$ for all $s\in S$. Then we have inf $S\leq s\leq$ sup $S$ which implies that inf $S\leq$ sup $S$.
              \item [(b)] What can you say about $S$ if inf $S=$ sup $S$?\\
                    \textbf{Answer: } Again by definition, if inf $S=$ sup $S$, we must have inf $S=s=$ sup $S$ for all $s\in S$. Then there can only be a single element in $S$ which is also both the supremum and infimum of $S$.
          \end{itemize}
    \item [4.7] Let $S$ and $T$ be nonempty bounded subsets of $\mathbb{R}$.
          \begin{itemize}
              \item [(a)] Prove if $S\subseteq T$, then inf $T\leq$ inf $S\leq$ sup $S\leq$ sup $T$.\\
                    \textbf{Answer: } By definition of upper bound, sup $T\geq t$ for all $t\in T$. Then by definition of subset, $s\in T$ for all $s\in S$. Therefore sup $T$ is an upper bound of $S$, i.e. sup $T\geq s$ for all $s\in S$. We also have that sup $S\geq s$ for all $s\in S$, with the added requirement of being the least upper bound by definition of supremum. Therefore sup $S\leq$ sup $T$.\\
                    Similarly, since inf $T\leq t$ for all $t\in T$ and $s\in T$ for all $s\in S$, inf $T$ is a lower bound of $S$; therefore the greatest lower bound inf $S$ must be greater or equal to inf $T$, i.e. inf $T\leq$ inf $S$.\\
                    By 4.6(a), we also have inf $S\leq$ sup $S$. Therefore inf $T\leq$ inf $S\leq$ sup $S\leq$ sup $T$.
              \item [(b)] Prove sup$(S\cup T)=$max\{sup $S$, sup $T$\}\\
                    \textbf{Answer: } By cases sup $S\leq$ sup $T$ and sup $S>$ sup $T$.\\
                    Case sup $S\leq$ sup $T$: Then max\{sup $S$, sup $T$\}=sup $T$. By definition of union, $S\cup T$ contains all elements from $S$ or $T$. Since sup $T\geq$ sup $S$, sup $T\geq s$ for all $s\in S$. By definition of supremum, we also have sup $T\geq t$ for all $t\in T$. Therefore sup $T$ is the supremum of the union.\\
                    Case sup $S>$ sup $T$: We have max\{sup $S$, sup $T$\}=sup $S$. Similar to the previous case, sup $S\geq s$ for all $s\in S$ by definition of supremum. In addition, sup $S\geq t$ for all $t\in T$ since sup $S>$ sup $T$. Therefore sup $S$ is the supremum of the union.
          \end{itemize}
    \item [4.8] Let $S$ and $T$ be nonempty subsets of $\mathbb{R}$ with the following property: $s\leq t$ for all $s\in S$ and $t\in T$.
          \begin{itemize}
              \item [(a)] Observe $S$ is bounded above and $T$ is bounded below.\\
                    \textbf{Answer: } Since $s\leq t$ for all $s\in S$ and $t\in T$, we can select any $t\in T$ to be an upper bound of $S$ (there is at least one such $t$ since $T$ is nonempty). Similarly, we can also select any $s\in S$ to be a lower bound of $T$. Therefore $S$ is bounded above and $T$ is bounded below.
              \item [(b)] Prove sup $S\leq$ inf $T$.\\
                    \textbf{Answer: } By contradiction. Assume sup $S>$ inf $T$. As shown in part (a), any $t\in T$ is an upper bound of $S$. Therefore the least upper bound sup $S\leq t$ for all $t\in T$, in other words, sup $S$ is a lower bound of $T$. However, since sup $S>$ inf $T$, inf $T$ cannot be the greatest lower bound of $T$. Therefore our assumption is false and sup $S\leq$ inf $T$.
              \item [(c)] Give an example of such sets $S$ and $T$ where $S\cap T$ is nonempty.\\
                    \textbf{Answer: } $S=[-1,0], T=[0,1]$.
              \item [(d)] Give an example of sets $S$ and $T$ where sup $S=$ inf $T$ and $S\cap T$ is the empty set.\\
                    \textbf{Answer: } $S=[-1,0), T=(0,1]$.
          \end{itemize}
    \item [4.14] Let $A$ and $B$ be nonempty bounded subsets of $\mathbb{R}$, and let $A+B$ be the set of all sums $a+b$ where $a\in A$ and $b\in B$.
          \begin{itemize}
              \item [(a)] Prove sup$(A+B)=$ sup $A+$sup $B$.\\
                    \textbf{Answer: } By definition of supremum, sup $A\geq a$ and sup $B\geq b$ for every $a\in A, b\in B$. Then, sup $A+$sup $B\geq a+b$. Since $a+b$ is also an arbitrary member of the set $A+B$, sup $A+$sup $B$ is an upper bound of the set. Then, sup$(A+B)\leq$ sup $A+$sup $B$ since it is the least upper bound of the set $A+B$. \\
                    We now will show that sup$(A+B)\nless$ sup $A+$sup $B$ by contradiction. Suppose sup$(A+B)<$ sup $A+$sup $B$, then there must exist an $r\in\mathbb{Q}$ such that sup$(A+B)<r<$ sup $A+$sup $B$ since $\mathbb{Q}$ is dense in $\mathbb{R}$. Then, since sup $A\geq a$ and sup $B\geq b$ for every $a\in A, b\in B$, $r<a+b$. Therefore $r$ is also a member of the set $A+B$ while being greater than the supremum of the set sup$(A+B)$, which leads to a contradiction. Then the only possibility is that sup$(A+B)=$ sup $A+$sup $B$.
              \item [(b)] Prove inf$(A+B)=$ inf $A+$inf $B$.\\
                    \textbf{Answer: } Similar to the part (a), Since inf $A\leq a$ and inf $B\leq b$ for every $a\in A, b\in B$, inf $A$+inf $B\leq a+b$. Then inf $A$+inf $B$ is a lower bound of the set $A+B$ and as a result inf $A$+inf $B$ must be less or equal to the greatest lower bound inf$(A+B)$, i.e. inf $A$+inf $B\leq$ inf$(A+B)$.\\
                    We will show that inf $A$+inf $B\nless$ inf$(A+B)$ by contradiction. Suppose inf $A$+inf $B<$ inf$(A+B)$, then there must exist an $r\in\mathbb{Q}$ such that inf $A$+inf $B<r<$ inf$(A+B)$. Since inf $A\leq a$ and inf $B\leq b$ for every $a\in A, b\in B$, $r>a+b$. Then $r$ is a member of the set $A+B$ while being less than the infimum inf$(A+B)$ which contradicts. Therefore inf$(A+B)=$ inf $A+$inf $B$.
          \end{itemize}
    \item [4.15] Let $a,b\in\mathbb{R}$. Show if $a\leq b+\dfrac{1}{n}$ for all $n\in\mathbb{N}$, then $a\leq b$. Compare Exercise 3.8.\\
          \textbf{Answer: } By contradiction. Suppose there exists $a,b\in\mathbb{R}$ such that $a\leq b+\dfrac{1}{n}$ for all $n\in\mathbb{N}$ and $a>b$. Since $a>b$, we have $a-b>0$. Then by the Archimedean property, there exists an $n\in\mathbb{N}$ that can scale it past $1$, i.e. $n(a-b)>1$. Upon dividing both sides by $n$ we have $a-b>\dfrac{1}{n}\implies a>b+\dfrac{1}{n}$, which contradicts our assumption. Therefore if $a\leq b+\dfrac{1}{n}$ for all $n\in\mathbb{N}$, then $a\leq b$.
    \item [4.16] Show sup\{$r\in\mathbb{Q}:r<a$\}$=a$ for each $a\in\mathbb{R}$.\\
          \textbf{Answer: } Since $r<a$ for all $r$ in the set, $a$ is automatically an upper bound. To show that $a$ is the supremum of the set, we will show that it is the least upper bound by contradiction. Suppose there exists another upper bound $b\in\mathbb{R}$ such that $b<a$. Then, there exists an $r\in\mathbb{Q}$ such that $b<r<a$ since $\mathbb{Q}$ is dense in $\mathbb{R}$. However, such $r$ would be be in the set \{$r\in\mathbb{Q}:r<a$\} and $b<r$ implies that $b$ is not an upper bound of the set, which contradicts our assumption. Therefore sup\{$r\in\mathbb{Q}:r<a$\}$=a$ for each $a\in\mathbb{R}$.
    \item [P1] Write down the converse and the contrapositive of the following statement regarding a real number $x$: \[\text{If } x>0, \text{then } x^2-x>0.\] Then determine which (if any) of the three statements are true for all real numbers $x$.\\
          \textbf{Answer: }\\Converse: if $x^2-x>0$, then $x>0$, which is false by counterexample $x=-1$. \\Contrapositive: if $x^2-x\leq0$, then $x\leq0$, which is false by counterexample $x=1$.
    \item [P2] Prove that $\sqrt{3}$ is not rational.\\
          \textbf{Answer: } By contradiction. Suppose $\sqrt{3}$ is rational, then by definition of rational numbers, there must exist $p,q\in\mathbb{Z}$ such that $\dfrac{p}{q}=\sqrt{3}$, where $p,q$ have no common factors upon simplying. Then, we also have $\dfrac{p^2}{q^2}=3\implies p^2=3q^2$. Since $p,q\in\mathbb{Z}$ and by extension $p^2,q^2\in\mathbb{Z}$, $p^2\mid 3$. Additionally, $p\mid 3$ because $\sqrt{3}\notin\mathbb{Z}$. Then, $p^2\mid 9$. By substituting $p^2=3q^2$, we now have $3q^2\mid 9\implies q^2\mid 3$, implying $q\mid 3$ by previous logic. Then $p,q$ have common factor $3$ which contradicts with our initial assumption. Therefore $\sqrt{3}$ is not rational.
\end{itemize}

\end{document}