\documentclass{article}
\usepackage[margin=1in]{geometry}
\usepackage{setspace}
\usepackage{amsmath}
\usepackage{amssymb}
\usepackage{physics}

\title{Math 131A Midterm 2}
\author{Jiaping Zeng}
\date{9/1/2020}

\begin{document}
\setstretch{1.5}
\begin{itemize}
    \item [1.] Let $\sum_{n=1}^\infty a_n$ be a convergent series. Prove that for every $\epsilon>0$ there exists $N\in\mathbb{N}$ such that $\sum_{n=N+1}^\infty a_n$ is convergent and \[\abs{\sum_{n=N+1}^\infty a_n}<\epsilon.\]
          \textbf{Proof: } Since $\sum_{n=N+1}^\infty a_n$ is a subsequence of a converging sequence of partial sums (definition of series), it must also converge by Theorem 11.3. In addition, $\sum_{n=1}^\infty a_n$ must satisfy the Cauchy criterion, i.e. for $\epsilon>0$ there exists an $N$ such that $n\geq m>N\implies\abs{\sum_{k=m}^na_k}<\epsilon$. We can take $n=\infty$ and $m=N+1$ which gives us $\abs{\sum_{k=N+1}^\infty a_k}<\epsilon$.
\end{itemize}
\newpage
\begin{itemize}
    \item [2.] Let $(a_n)_{n\in\mathbb{N}}$ be the sequence defined by \[a_n=\begin{cases}\left(\frac{n^2-2}{6n^2+n}\right)^n & \text{if } n=2k-1\text{ for some }k\in\mathbb{N} \\\frac{4n}{5^n} & \text{if } n=2k\text{ for some }k\in\mathbb{N}.\\\end{cases}\] Does the series $\sum_{n=1}^\infty a_n$ converge? Justify your answer.\\
          \textbf{Claim: } The series $\sum_{n=1}^\infty a_n$ does converge.\\
          \textbf{Proof: } We can start by checking each part of the piecewise definition:\\$\sum\left(\frac{n^2-2}{6n^2+n}\right)^n\implies\sqrt[n]{\abs{a_n}}=\frac{n^2-2}{6n+n}\rightarrow\frac{1}{6}<1$, therefore converges by root test.\\$\sum\frac{4n}{5^n}\implies\abs{\frac{a_{n+1}}{a_n}}=\frac{4(n+1)}{5^{n+1}}\cdot\frac{5^n}{4n}=\frac{4n+4}{20n}\rightarrow\frac{1}{5}<1$, therefore converges by ratio test.\\Then, we can rewrite the piecewise definition into a sum of two series (in terms of $k$) as follows:\\\[\sum_{n=1}^\infty a_n=\sum_{k=1}^\infty \left(\frac{(2k-1)^2-2}{6(2k-1)^2+(2k-1)}\right)^{2k-1}+\sum_{k=1}^\infty\frac{8k}{5^{2k}}\]\\Since $\left(\frac{n^2-2}{6n^2+n}\right)^n>0$ for $n\geq 2$, removing the even terms (which are all positive) results in a smaller sum overall, i.e. $\sum_{k=1}^\infty\left(\frac{(2k-1)^2-2}{6(2k-1)^2+(2k-1)}\right)^{2k-1}\leq\sum_{n=1}^\infty\left(\frac{n^2-2}{6n^2+n}\right)^n$. Then, since $\sum_{n=1}^\infty\left(\frac{n^2-2}{6n^2+n}\right)^n$ converges, $\sum_{k=1}^\infty\left(\frac{(2k-1)^2-2}{6(2k-1)^2+(2k-1)}\right)^{2k-1}$ must converge as well by comparison test.\\Similarly, since $\frac{4n}{5^n}>0$ for $n\geq 1$, then removing the odd terms results in a smaller sum, i.e. $\sum_{k=1}^\infty\frac{8k}{5^{2k}}\leq\sum\frac{4n}{5^n}$ which means that $\sum_{k=1}^\infty\frac{8k}{5^{2k}}$ also converges.\\Therefore $\sum_{n=1}^\infty a_n$ is a sum of two converging series and must also converge by limit sum law.
\end{itemize}
\newpage
\begin{itemize}
    \item [3.] Let $(s_n)$ and $(t_n)$ be sequences with $\text{lim sup }s_n$ and $\text{lim inf }t_n$ finite. Suppose $\text{lim inf }t_n=(\text{lim sup }s_n)+2$. Prove that there is $N$ such that $t_n-s_n>1$ for all $n>N$.\\
          \textbf{Proof: } By definition, $\text{lim sup }s_n=\text{lim}_{N\rightarrow\infty}\text{sup}\{s_n:n>N\}$. Let $v_N=\text{sup}\{s_n:n>N\}$ and $s=\text{lim}_{N\rightarrow\infty}v_N=\text{lim sup }s_n$. Then by definition of limit, pick $\epsilon=\frac{1}{2}>0$, there exists an $N_s$ such that $N>N_s\implies\abs{v_N-s}<\frac{1}{2}\implies -\frac{1}{2}+s<v_N<\frac{1}{2}+s$. By definition of supremum, $v_N\geq s_n$ for $n>N$. Combining the inequalities together we have $N>N_s\implies s_n\leq v_N<\frac{1}{2}+s$.\\
          Similarly, let $u_N=\text{inf}\{t_n:n>N\}$ and $t=\text{lim}_{N\rightarrow\infty}u_N=\text{lim inf }t_n$. Then there exists an $N_t$ such that $N>N_t\implies\abs{u_N-t}<\frac{1}{2}\implies -\frac{1}{2}+t<u_N<\frac{1}{2}+t$. By definition of infimum, $u_N\leq t_n$ for $n>N$. Then we have $N>N_t\implies -\frac{1}{2}+t<u_N\leq t_n$.\\
          Then if we take $N>\text{max}\{N_s,N_t\}$, we have both $s_n<\frac{1}{2}+s$ and $-\frac{1}{2}+t<t_n$. In addition, using the given assumption $\text{lim inf }t_n=(\text{lim sup }s_n)+2$, we have $t=s+2$. Then by substitution $-\frac{1}{2}+s+2<t_n\implies\frac{1}{2}+s<t_n-1$ which when combined with $s_n<\frac{1}{2}+s$ gives us $s_n<\frac{1}{2}+s<t_n-1\implies t_n-s_n>1$.
\end{itemize}
\newpage
\begin{itemize}
    \item [4.] Let $f$ and $g$ be real-valued functions defined on all of $\mathbb{R}$, and let $x_0\in\mathbb{R}$. Suppose both $f$ and $g$ are continuous at the point $x_0$, and that $f(x)\leq g(x)$ for all $x<x_0$. Prove that $f(x_0)\leq g(x_0)$.\\
          \textbf{Proof: } Since $\mathbb{Q}$ is dense in $\mathbb{R}$, there exists a sequence of rational numbers $(r_n)$ that converges to $x_0$ where $r_n<x_0$ for all $n$. Furthermore, since $f(x)\leq g(x)$ for all $x<x_0$, we have $f(r_n)\leq g(r_n)$ for all $n$, which implies that $\text{lim}\,f(r_n)\leq\text{lim}\,g(r_n)$. Then, since $f$ is continuous at $x_0$, we have $\text{lim}\,f(r_n)=f(x_0)$ by definition of continuity. Similarly, $\text{lim}\,g(r_n)=g(x_0)$. Connecting the previous equations and inequality gives us $f(x_0)=\text{lim}\,f(r_n)\leq\text{lim}\,g(r_n)=g(x_0)$, therefore $f(x_0)\leq g(x_0)$.
\end{itemize}
\newpage
\begin{itemize}
    \item [5.] Let $(s_n)$ be a bounded sequence in $\mathbb{R}$. Suppose that the set \[\{n\in\mathbb{N}:s_n\leq 0\}\] is infinite. Prove there is a subsequence $(t_k)$ of $(s_n)$ which is convergent and has $\text{lim }t_k\leq 0$.\\
          \textbf{Proof: } Let $\epsilon>0$, then $s_n\leq 0\implies s_n+\epsilon\leq\epsilon$. We can take $t=-\epsilon<0$ which gives us $s_n-t=s_n+\epsilon<\epsilon\implies\abs{s_n-t}\leq\epsilon$. Then the set $\{n\in\mathbb{N}:\abs{s_n-t}\leq\epsilon\}$ is infinite; by Theorem 11.2(i) there exists a subsequence $(t_k)$ of $(s_n)$ converging to $t<0$ as defined, i.e. $\text{lim }t_k\leq 0$.
\end{itemize}
\end{document}