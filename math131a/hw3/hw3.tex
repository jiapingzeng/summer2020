\documentclass{article}
\usepackage[margin=1in]{geometry}
\usepackage{enumitem}
\usepackage{setspace}
\usepackage{amsmath}
\usepackage{amssymb}
\usepackage{physics}

\title{Math 131A Homework 3}
\date{8/20/2020}
\author{Jiaping Zeng}

\begin{document}
\setstretch{1.35}
\maketitle

\begin{itemize}
    \item [10.9] Let $s_1=1$ and $s_{n+1}=(\dfrac{n}{n+1})s_n^2$ for $n\geq 1$.
          \begin{itemize}
              \item [(a)] Find $s_2$, $s_3$ and $s_4$.\\
                    $s_2=\dfrac{1}{2}\cdot 1=\dfrac{1}{2}$\\$s_3=\dfrac{2}{3}\cdot\dfrac{1}{4}=\dfrac{1}{6}$\\$s_3=\dfrac{3}{4}\cdot\dfrac{1}{36}=\dfrac{1}{48}$
              \item [(b)] Show $\text{lim }s_n$ exists.\\
                    Since $\dfrac{n}{n+1}<1$ and $s_n^2<s_n$ for all $n\geq 1$, $\dfrac{n}{n+1}s_n^2<1\cdot s_n\implies s_{n+1}\leq s_n$ for all $n\geq 1$. Therefore $(s_n)$ is monotone and decreasing. Then, since $(s_n)$ is decreasing, $s_1$ is an upper bound of the set and therefore $(s_n)$ is bounded above. In addition, $s_{n+1}$ is never negative because $n\leq 1\implies\dfrac{n}{n+1}>0$ and $s_n^2\geq 0$. Therefore $0$ is an lower bound of the set and $(s_n)$ is bounded below. Then $\text{lim }s_n$ exists by Theorem 10.2.
              \item [(c)] Prove $\text{lim }s_n=0$.\\
                    Let $s=\text{lim }s_n$, then $s=\text{lim }s_{n+1}=\text{lim }(\dfrac{n}{n+1})s_n^2=\text{lim }(\dfrac{n}{n+1})\cdot\text{lim }s_n^2=\text{lim }s_n^2=s^2\implies s=s^2$. Then we have $s=0$ or $s=1$. However, since $s_1=1$ is an upper bound and $(s_n)$ is strictly decreasing as shown in part (b), $s\neq 1$. Therefore $s=0=\text{lim }s_n$.
          \end{itemize}
    \item [11.2] Consider the sequences defined as follows:\[a_n=(-1)^n,\;b_n=\dfrac{1}{n},\; c_n=n^2,\;d_n=\dfrac{6n+4}{7n-3}.\]
          \begin{itemize}
              \item [(a)] For each sequence, give an example of a monotone subsequence.
              \item [(b)] For each sequence, give its set of subsequential limits.
              \item [(c)] For each sequence, give its lim sup and lim inf.
              \item [(d)] Which of the sequences converge? diverge to $+\infty$? diverge to $-\infty$?
              \item [(e)] Which of the sequences is bounded?
          \end{itemize}
    \item [11.5] Let $(q_n)$ be an enumeration of all the rationals in the interval $(0,1]$.
          \begin{itemize}
              \item [(a)] Give the set of subsequential limits for $(q_n)$.
              \item [(b)] Give the values of $\text{lim sup }q_n$ and $\text{lim inf }q_n$.
          \end{itemize}
    \item [12.2] Prove $\text{lim sup }\abs{s_n}=0$ if and only if $\text{lim }s_n=0$.
    \item [12.4] Show $\text{lim sup}(s_n+t_n)\leq\text{lim sup} s_n+\text{lim sup} t_n$ for bounded sequences $(s_n)$ and $(t_n)$.
    \item [12.9] \begin{itemize}
              \item [(a)] Prove that if $\text{lim }s_n=\infty$ and $\text{lim inf}t_n>0$, then $\text{lim }s_nt_n=+\infty$.
              \item [(b)] Prove that if $\text{lim sup }s_n=+\infty$ and $\text{lim inf }t_n>0$, then $\text{lim sup }s_nt_n=+\infty$.
              \item [(c)] Observe that Exercise 12.7 is the special case of (b) where $t_n=k$ for all $n\in\mathbb{N}$.
          \end{itemize}
    \item [12.10] Prove $(s_n)$ is bounded if and only if $\text{lim sup }\abs{s_n}<+\infty$.
    \item [12.12] Let $(s_n)$ be a sequence of nonnegative numbers, and for each $n$ define $\sigma_n=\frac{1}{n}(s_1+s_2+\ldots+s_n)$.
          \begin{itemize}
              \item [(a)] Show \[\text{lim inf }s_n\leq\text{lim inf }\sigma_n\leq\text{lim sup }\sigma_n\leq\text{lim sup }s_n.\]
              \item [(b)] Show that if $\text{lim }s_n$ exists, then $\text{lim }\sigma_n$ exists and $\text{lim }\sigma_n=\text{lim }s_n$.
              \item [(c)] Give an example where $\text{lim }\sigma_n$ exists, but $\text{lim }s_n$ does not exist.
          \end{itemize}
    \item [14.5] Suppose $\sum a_n=A$ and $\sum b_n=B$ where $A$ and $B$ are real numbers. Use limit theorems to quickly prove the following.
          \begin{itemize}
              \item [(a)] $\sum(a_n+b_n)=A+B$.
              \text{Proof: } 
              \item [(b)] $\sum ka_n=kA$ for $k\in\mathbb{R}$.
              \item [(c)] Is $\sum a_nb_n=AB$ a reasonable conjuecture? Discuss.
          \end{itemize}
    \item [P1] Let $(s_n)$ be the sequence \[s_n=\dfrac{n^2+1}{n^2+2n}\text{ sin }n.\] Prove that $(s_n)$ has a convergent subsequence.
    \item [P2] Let $(s_n)$ be a sequence that contains eveyr integer. Prove that there is a subsequence of $(s_n)$ which diverges to $-\infty$.
    \item [P3] Suppose $(s_n)$ is a sequence and $(t_k)$ is a subsequence of $(s_n)$ such that $(t_k)$ converges. Prove that $\text{lim }t_k\leq\text{lim sup }s_n$.
    \item [P4] For each series, determine whether the series $(1)$ converges to a real number, $(2)$ diverges to $+\infty$, $(3)$ diverges to $-\infty$, or $(4)$ none of these. Prove your answers.
          \begin{itemize}
              \item [(a)] $\sum_{n=1}^\infty \frac{\text{cos}^2(n)}{n^2}$
              \item [(b)] $\sum_{n=1}^\infty \frac{n-1}{n^2}$
              \item [(c)] $\sum_{n=1}^\infty (-1)^n$
              \item [(d)] $\sum_{n=1}^\infty \frac{n+1}{n^3-1}$
          \end{itemize}
\end{itemize}
\end{document}