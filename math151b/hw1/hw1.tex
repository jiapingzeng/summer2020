\documentclass{article}
\usepackage[margin=1in]{geometry}
\usepackage{enumitem}
\usepackage{setspace}
\usepackage{amsmath}
\usepackage{amssymb}

\title{Math 131A Homework 1}
\date{8/3/2020}
\author{Jiaping Zeng}

\begin{document}
\setstretch{1.35}
\maketitle

\begin{itemize}
    \item [Q1]
          \begin{itemize}
              \item [(a)] Continuity: Since $f(t,y)=\dfrac{1+y}{t}$ is only undefined at $t=0$ and the given interval $t\in[1,2]$ does not include $0$, $f(t,y)$ is continuous on the given interval.\\Lipschitz: $\dfrac{\delta f(t,y)}{\delta y}=\dfrac{1}{t}\leq\dfrac{1}{1}=1=L$.\\Therefore this IVP is well-posed by Theorem 5.6.
              \item [(b)] Continuity: $f(t,y)=y\,cos(t)$ is continuous on $t\in[0,1]$ because both $y$ and $cos(t)$ are defined everywhere.\\Lipschitz: $\dfrac{\delta f(t,y)}{\delta y}=cos(t)\leq cos(0)=1=L$.\\Therefore this IVP is well-posed by Theorem 5.6.
          \end{itemize}
    \item [Q2]
          \begin{itemize}
              \item [(a)] $y(1.5)\approx 2+0.5\cdot\dfrac{1+1}{1+2}=\dfrac{7}{3}\approx 2.333333$\\$y(2)\approx \dfrac{7}{3}+0.5\cdot\dfrac{1+1.5}{1+\frac{7}{3}}=\boxed{\dfrac{65}{24}}\approx 2.708333$
              \item [(b)]
                    \begin{tabular}{|c|c|}
                        \hline
                        h    & y(2): Euler        \\
                        \hline
                        0.5  & 2.7083333333333335 \\
                        \hline
                        0.2  & 2.729166194327493  \\
                        \hline
                        0.1  & 2.7355407599225927 \\
                        \hline
                        0.01 & 2.741056919124695  \\
                        \hline
                    \end{tabular}
                    \begin{verbatim}
def q2b():
    h_vals = [0.5, 0.2, 0.1, 0.01]  # step sizes
    t_0, t_f, y_0 = 1, 2, 2  # initial/final values

    for h in h_vals:
        t, y, steps = t_0, y_0, int((t_f-t_0)/h)
        for _ in range(steps):
            y_prev = y
            y = y_prev + h*(1+t)/(1+y_prev)
            t += h
        print(f"h={h}, y={y}")
                    \end{verbatim}
              \item [(c)] Exact solution: $y(2)=\sqrt{14}-1\approx 2.741657387$; we can see that, as expected, smaller $h$ values resulted in approximations that are closer to the exact $y(2)$ value.
          \end{itemize}
    \item [Q3]
          \begin{itemize}
              \item [(a)] $\dfrac{df(t,y)}{dt}=\dfrac{\delta f(t,y)}{\delta t}+\dfrac{\delta f(t,y)}{\delta y}\cdot\dfrac{\delta y}{\delta t}=-y^2e^{-t}+2ye^{-t}\cdot y^2e^{-t}=\boxed{-y^2e^{-t}+2y^3e^{-2t}}$
              \item [(b)] $y(0.5)\approx 1+0.5\cdot(1^2\cdot e^0)+0.125\cdot(-1^2\cdot e^0+2\cdot 1^3\cdot e^0)=1+0.5+0.125=1.625$\\$y(1)\approx 1.625+0.5\cdot(1.625^2\cdot e^{-0.5})+0.125\cdot(-1.625^2\cdot e^{-0.5}+2\cdot 1.625^2\cdot e^{-1})\approx\boxed{2.620252}$
              \item [(c)]
                    \begin{tabular}{|c|c|c|}
                        \hline
                        h    & y(1): Euler        & y(1): Taylor-O2    \\
                        \hline
                        0.5  & 2.1823469921767127 & 2.620251616284609  \\
                        \hline
                        0.1  & 2.5318870490201966 & 2.711460067794978  \\
                        \hline
                        0.01 & 2.6955194578645645 & 2.7182049559385204 \\
                        \hline
                    \end{tabular}
                    \begin{verbatim}
def q3c():
    h_vals = [0.5, 0.1, 0.01]  # step sizes
    t_0, t_f, y_0 = 0, 1, 1  # initial/final values

    print("(Euler's method)")
    for h in h_vals:
        t, y, steps = t_0, y_0, int((t_f-t_0)/h)
        for _ in range(steps):
            y_prev = y
            y = y_prev + h*pow(y_prev, 2)*exp(-t)
            t += h
        print(f"h={h}, y={y}")
                    
    print("(Taylor method)")
    for h in h_vals:
        t, y, steps = t_0, y_0, int((t_f-t_0)/h)
        for _ in range(steps):
            y_prev = y
            y = y_prev + h*pow(y_prev, 2)*exp(-t) + pow(h, 2) * \
                (-1*pow(y_prev, 2)*exp(-t)+2*pow(y_prev, 3)*exp(-2*t))/2
            t += h
        print(f"h={h}, y={y}")
                    \end{verbatim}
              \item [(d)]
                    \begin{tabular}{|c|c|c|}
                        \hline
                        h    & Error: Euler         & Error: Taylor-O2      \\
                        \hline
                        0.5  & 0.5359348362823324   & 0.0980302121744363    \\
                        \hline
                        0.1  & 0.18639477943884852  & 0.006821760664067256  \\
                        \hline
                        0.01 & 0.022762370594480608 & 7.687252052468452e-05 \\
                        \hline
                    \end{tabular}\\\\
                    As shown in the table above, the error for both methods decreases as $h$ decreases. In addition, Taylor method of order 2 converges much faster than Euler's method as expected.
          \end{itemize}
\end{itemize}

\end{document}